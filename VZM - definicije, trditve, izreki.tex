\documentclass[11pt]{article}
\usepackage[utf8]{inputenc}
\usepackage[slovene]{babel}

\usepackage{amsthm}
\usepackage{amsmath, amssymb, amsfonts}
\usepackage{relsize}
\usepackage{mathrsfs}

\newcommand{\R}{\mathbb{R}}
\newcommand{\N}{\mathbb{N}}
\renewcommand{\P}{\mathbb{P}}
\newcommand{\E}{\mathbb{E}}
\newcommand{\A}{\mathcal{A}}
\newcommand{\B}{\mathcal{B}}
\newcommand{\F}{\mathcal{F}}
\renewcommand{\c}{\mathsf{c}}
\newcommand{\diff}{\overset{\text{def}}{\iff}}
\newcommand{\imp}{~\Rightarrow~}
\newcommand{\set}[1]{\{#1\}}
\newcommand{\oklepaj}[1]{\left(#1\right)}

\theoremstyle{definition}
\newtheorem{definicija}{Definicija}[section]

\theoremstyle{definition}
\newtheorem{trditev}{Trditev}[section]

\theoremstyle{definition}
\newtheorem{izrek}{Izrek}[section]

\theoremstyle{definition}
\newtheorem{metoda}{Metoda}[section]

\newtheorem*{posledica}{Posledica}
\newtheorem*{opomba}{Opomba}
\newtheorem*{komentar}{Komentar}
\newtheorem{lema}{Lema}
\newtheorem*{dokaz}{Dokaz}
\newtheorem*{posplošitev}{Posplošitev}
\newtheorem*{dogovor}{Dogovor}
\newtheorem*{sklep}{Sklep}

\title{Verjetnost z mero - definicije, trditve in izreki}
\author{Oskar Vavtar \\
po predavanjih profesorja Matija Vidmarja}
\date{2021/22}

\begin{document}
\maketitle
\pagebreak
\tableofcontents
\pagebreak

% #################################################################################################

\section{Merljivost in mera}
\vspace{0.5cm}

% *************************************************************************************************

\subsection{Merljive množice}
\vspace{0.5cm}

\begin{definicija}

Naj bo $\A \subset 2^\Omega$ (t.j. $\A \in 2^{2^\Omega}$). Potem rečemo, da je $\A$ \textit{zaprta} za:
\begin{itemize}

\item $\c^\Omega$ (t.j. za komplement v $\Omega$) 
$$\diff ~~~\forall A: (A \in \Omega \imp \Omega \setminus A \in \A);$$

\item $\cap$ (t.j. za preseke)
$$\diff ~~~A_1 \cap A_2 \in \A ~~\text{brž ko je}~ \set{A_1,A_2} \subset A;$$

\item $\cup$ (t.j. za unije)
$$\diff ~~~A_1 \cup A_2 \in \A ~~\text{brž ko je}~ \set{A_1,A_2} \subset A;$$

\item $\setminus$ (t.j. za razlike)
$$\diff ~~~A_1 \setminus A_2 \in \A ~~\text{brž ko je}~ \set{A_1,A_2} \subset A;$$

\item $\sigma\cap$ (t.j. za števne preseke)
$$\diff ~~~\bigcap_{n\in\N} A_n \in \A ~~\text{za vsako zaporedje}~ (A_n)_{n\in\N} ~\text{iz}~ \A;$$

\item $\sigma\cup$ (t.j. za števne unije)
$$\diff ~~~\bigcup_{n\in\N} A_n \in \A ~~\text{za vsako zaporedje}~ (A_n)_{n\in\N} ~\text{iz}~ \A.$$

\end{itemize}

\end{definicija}
\vspace{0.5cm}

\begin{definicija}

$\A$ je \textit{$\sigma$-algebra} na $\Omega$
\begin{align*}
&\diff ~~~(\Omega,\A) ~~\text{je merljiv prostor} \\
&\diff ~~~\emptyset \in \A ~\text{in}~ \A ~\text{je zaprt za}~ \c^\Omega ~\text{in za}~ \sigma\cup.
\end{align*}
V primeru, da $\A$ je $\sigma$-algebra na $\Omega$:
\begin{itemize}
	\item $A$ je $\A$-merljiva $~\diff~ A \in \A;$
	\item $\B$ je pod-$\sigma$-algebra $~\diff~ \B$ je $\sigma$-algebra na $\Omega$ in $\B \subset \A.$
\end{itemize}

\end{definicija}
\vspace{0.5cm}

\begin{trditev}

Naj bo $\A \subset 2^\Omega$ zaprta za $\c^\Omega$ in naj bo $\emptyset \in \A$. Potem je $\A$ $\sigma$-algebra na $\Omega$, če je $\A$ zaprta za števne preseke, in v tem primeru je $\A$ zaprta za $\cap$, $\cup$ in $\setminus$.

\end{trditev}
\vspace{0.5cm}

% *************************************************************************************************

\subsection{Mere}
\vspace{0.5cm}

\begin{definicija}

Naj bo $(\Omega,\F)$ merljiv prostor in $\mu: \F \rightarrow [0,\infty]$. $\mu$ je \textit{mera} na $(\Omega,\F)$ $\diff$
\begin{itemize}
	\item $\mu(\emptyset) ~=~ 0;$
	\item $\mu\mathlarger{\oklepaj{\bigcup_{n\in\N}} ~=~ \sum_{n\in\N} \mu(A_n)}$ za vsako zaporedje $(A_n)_{n\in\N}$ iz $\F$, ki sestoji iz paroma disjunktnih dogodkov.
\end{itemize}
Lastnosti:
\begin{itemize}
	\item Mera $\mu$ na $(\Omega,\F)$ je \textit{končna} $\diff$ $\mu(\Omega)<\infty$.
	\item Mera $\mu$ na $(\Omega,\F)$ je \textit{verjetnostna}\footnote{Tudi: $\mu$ je \textit{verjetnost}.} $\diff$ $\mu(\Omega) = 1$
	
	\item Mera $\mu$ na $(\Omega,\F)$ je \textit{$\sigma$-končna} $\diff$ obstaja zaporedje $(A_n)_{n\in\N}$ v $\F$, da je
	\begin{align*}
	\bigcup_{n\in\N} ~&=~ \Omega ~~~\text{in} \\
	\mu(A_n) ~&<~ \infty, ~~~\forall n \in \N
	\end{align*}
	
\end{itemize}
$(\Omega,\F,\mu)$ je prostor z mero $\diff$ $\mu$ je mera na $(\Omega,\F)$. Če je $\mu$ mera na $(\Omega,\F)$ potem je $\mu(\Omega)$ \textit{masa} mere $\mu$. Če je $A \in \F$, potem je:
\begin{itemize}
	\item $A$ je \textit{$\mu$-zanemarljiv} $\diff$ $\mu(A) = 0$;
	\item $A$ je \textit{$\mu$-trivialna} $\diff$ $A$ ali $\Omega\setminus A$ je $\mu$-zanemarljiva
\end{itemize}
Če imamo poleg tega še lastnost $P(\omega)$ v $\omega \in A$, potem
\begin{itemize}
	\item $P(\omega)$ drži \textit{$\mu$-skoraj povsod} ($\mu$-s.p.) v $\omega \in A$ $\diff$ 
	$$A_{\neg P} ~:=~ \set{\omega \in \Omega \mid \neg P(\omega) \in \F ~\text{in}~ \mu(A_{\neg P}) = 0};$$
	\item $P(\omega)$ drži \textit{$\mu$-skoraj gotovo} ($\mu$-s.g.) $\diff$ $P(\omega)$ drži $\mu$-skoraj povsod in $\mu$ je verjetnostna.
\end{itemize}
$P$ drži $\mu$-skoraj povsod na $A$ $\diff$ $P(\omega)$ drži $\mu$-skoraj povsod v $\omega \in A$. Podobno za ostale.

\end{definicija}
\vspace{0.5cm}

% *************************************************************************************************

\pagebreak

% #################################################################################################

\end{document}