\documentclass[11pt]{article}
\usepackage[utf8]{inputenc}
\usepackage[slovene]{babel}

\usepackage{amsthm}
\usepackage{amsmath, amssymb, amsfonts}
\usepackage{relsize}
\usepackage{mathrsfs}
\usepackage{bbm}

\newcommand{\R}{\mathbb{R}}
\newcommand{\N}{\mathbb{N}}
\renewcommand{\P}{\mathbb{P}}
\newcommand{\E}{\mathbb{E}}
\newcommand{\A}{\mathcal{A}}
\newcommand{\B}{\mathcal{B}}
\newcommand{\F}{\mathcal{F}}
\newcommand{\G}{\mathcal{G}}
\renewcommand{\H}{\mathcal{H}}
\renewcommand{\c}{\mathsf{c}}
\newcommand{\diff}{\overset{\text{def}}{\iff}}
\newcommand{\imp}{~\Rightarrow~}
\newcommand{\set}[1]{\{#1\}}
\newcommand{\oklepaj}[1]{\left(#1\right)}
\newcommand{\1}{\mathbbm{1}}

\theoremstyle{definition}
\newtheorem{definicija}{Definicija}[section]

\theoremstyle{definition}
\newtheorem{trditev}{Trditev}[section]

\theoremstyle{definition}
\newtheorem{izrek}{Izrek}[section]

\theoremstyle{definition}
\newtheorem{metoda}{Metoda}[section]

\newtheorem*{posledica}{Posledica}
\newtheorem*{opomba}{Opomba}
\newtheorem*{komentar}{Komentar}
\newtheorem{lema}{Lema}
\newtheorem*{dokaz}{Dokaz}
\newtheorem*{posplošitev}{Posplošitev}
\newtheorem*{dogovor}{Dogovor}
\newtheorem*{sklep}{Sklep}

\title{Verjetnost z mero - definicije, trditve in izreki}
\author{Oskar Vavtar \\
po predavanjih profesorja Matija Vidmarja}
\date{2021/22}

\begin{document}
\maketitle
\pagebreak
\tableofcontents
\pagebreak

% #################################################################################################

\section{Merljivost in mera}
\vspace{0.5cm}

% *************************************************************************************************

\subsection{Merljive množice}
\vspace{0.5cm}

\begin{definicija}

Naj bo $\A \subset 2^\Omega$ (t.j. $\A \in 2^{2^\Omega}$). Potem rečemo, da je $\A$ \textit{zaprta} za:
\begin{itemize}

\item $\c^\Omega$ (t.j. za komplement v $\Omega$) 
$$\diff ~~~\forall A: (A \in \Omega \imp \Omega \setminus A \in \A);$$

\item $\cap$ (t.j. za preseke)
$$\diff ~~~A_1 \cap A_2 \in \A ~~\text{brž ko je}~ \set{A_1,A_2} \subset A;$$

\item $\cup$ (t.j. za unije)
$$\diff ~~~A_1 \cup A_2 \in \A ~~\text{brž ko je}~ \set{A_1,A_2} \subset A;$$

\item $\setminus$ (t.j. za razlike)
$$\diff ~~~A_1 \setminus A_2 \in \A ~~\text{brž ko je}~ \set{A_1,A_2} \subset A;$$

\item $\sigma\cap$ (t.j. za števne preseke)
$$\diff ~~~\bigcap_{n\in\N} A_n \in \A ~~\text{za vsako zaporedje}~ (A_n)_{n\in\N} ~\text{iz}~ \A;$$

\item $\sigma\cup$ (t.j. za števne unije)
$$\diff ~~~\bigcup_{n\in\N} A_n \in \A ~~\text{za vsako zaporedje}~ (A_n)_{n\in\N} ~\text{iz}~ \A.$$

\end{itemize}

\end{definicija}
\vspace{0.5cm}

\begin{definicija}

$\A$ je \textit{$\sigma$-algebra} na $\Omega$
\begin{align*}
&\diff ~~~(\Omega,\A) ~~\text{je merljiv prostor} \\
&\diff ~~~\emptyset \in \A ~\text{in}~ \A ~\text{je zaprt za}~ \c^\Omega ~\text{in za}~ \sigma\cup.
\end{align*}
V primeru, da $\A$ je $\sigma$-algebra na $\Omega$:
\begin{itemize}
	\item $A$ je $\A$-merljiva $~\diff~ A \in \A;$
	\item $\B$ je pod-$\sigma$-algebra $~\diff~ \B$ je $\sigma$-algebra na $\Omega$ in $\B \subset \A.$
\end{itemize}

\end{definicija}
\vspace{0.5cm}

\begin{trditev}

Naj bo $\A \subset 2^\Omega$ zaprta za $\c^\Omega$ in naj bo $\emptyset \in \A$. Potem je $\A$ $\sigma$-algebra na $\Omega$, če je $\A$ zaprta za števne preseke, in v tem primeru je $\A$ zaprta za $\cap$, $\cup$ in $\setminus$.

\end{trditev}
\vspace{0.5cm}

% *************************************************************************************************

\subsection{Mere}
\vspace{0.5cm}

\begin{definicija}

Naj bo $(\Omega,\F)$ merljiv prostor in $\mu: \F \rightarrow [0,\infty]$. $\mu$ je \textit{mera} na $(\Omega,\F)$ $\diff$
\begin{itemize}
	\item $\mu(\emptyset) ~=~ 0;$
	\item $\mu\mathlarger{\oklepaj{\bigcup_{n\in\N}} ~=~ \sum_{n\in\N} \mu(A_n)}$ za vsako zaporedje $(A_n)_{n\in\N}$ iz $\F$, ki sestoji iz paroma disjunktnih dogodkov.
\end{itemize}
Lastnosti:
\begin{itemize}
	\item Mera $\mu$ na $(\Omega,\F)$ je \textit{končna} $\diff$ $\mu(\Omega)<\infty$.
	\item Mera $\mu$ na $(\Omega,\F)$ je \textit{verjetnostna}\footnote{Tudi: $\mu$ je \textit{verjetnost}.} $\diff$ $\mu(\Omega) = 1$
	
	\item Mera $\mu$ na $(\Omega,\F)$ je \textit{$\sigma$-končna} $\diff$ obstaja zaporedje $(A_n)_{n\in\N}$ v $\F$, da je
	\begin{align*}
	\bigcup_{n\in\N} ~&=~ \Omega ~~~\text{in} \\
	\mu(A_n) ~&<~ \infty, ~~~\forall n \in \N
	\end{align*}
	
\end{itemize}
$(\Omega,\F,\mu)$ je prostor z mero $\diff$ $\mu$ je mera na $(\Omega,\F)$. Če je $\mu$ mera na $(\Omega,\F)$ potem je $\mu(\Omega)$ \textit{masa} mere $\mu$. Če je $A \in \F$, potem je:
\begin{itemize}
	\item $A$ je \textit{$\mu$-zanemarljiv} $\diff$ $\mu(A) = 0$;
	\item $A$ je \textit{$\mu$-trivialna} $\diff$ $A$ ali $\Omega\setminus A$ je $\mu$-zanemarljiva
\end{itemize}
Če imamo poleg tega še lastnost $P(\omega)$ v $\omega \in A$, potem
\begin{itemize}
	\item $P(\omega)$ drži \textit{$\mu$-skoraj povsod} ($\mu$-s.p.) v $\omega \in A$ $\diff$ 
	$$A_{\neg P} ~:=~ \set{\omega \in \Omega \mid \neg P(\omega) \in \F ~\text{in}~ \mu(A_{\neg P}) = 0};$$
	\item $P(\omega)$ drži \textit{$\mu$-skoraj gotovo} ($\mu$-s.g.) $\diff$ $P(\omega)$ drži $\mu$-skoraj povsod in $\mu$ je verjetnostna.
\end{itemize}
$P$ drži $\mu$-skoraj povsod na $A$ $\diff$ $P(\omega)$ drži $\mu$-skoraj povsod v $\omega \in A$. Podobno za ostale.

\end{definicija}
\vspace{0.5cm}

\begin{trditev}

Naj bo $\mu$ mera na $(\Omega, \F)$. Potem:
\begin{enumerate}

\item[(i)] $\mu$ je aditivna: 
$$\mu(A \cup B) ~=~ \mu(A) + \mu(B)$$
za vsaki disjunktivni množici $A,B \in \F$.

\item[(ii)] $\mu$ je monotona:
$$\mu(A) ~\leq~ \mu(B),$$
če je $A \subset B$ in $A,B \in \F$

\item[(iii)] $\mu$ je zvezna od spodaj:
$$\mu\oklepaj{\bigcup_{n \in \N} A_n} ~=~ {\uparrow\text{-}\lim_{n \rightarrow \infty}} \mu(A_n)$$
za vsako zaporedje $(A_n)_{n \in \N}$ iz $\F$, ki je nepadajoče glede na inkluzijo: $A_n \subset A_{n+1} ~\forall n \in \N$.

\item[(iv)] $\mu$ je števno subaditivna:
$$\mu\oklepaj{\bigcup_{n \in \N} A_n} ~\leq~ \sum_{n \in \N} \mu(A_n)$$
za vsako zaporedje $(A_n)_{n \in \N}$ iz $\F$.

\item[(v)] Naj bo $\mu$ končna:
$$\mu(\Omega \setminus A) ~=~ \mu(\Omega) - \mu(A) ~\forall A \in \F.$$
Naprej, $\mu$ je zvezna od zgoraj:
$$\mu\oklepaj{\bigcap_{n \in \N} A_n} ~=~ {\downarrow\text{-}\lim_{n \rightarrow \infty}}\mu(A_n)$$
za vsako zaporedje $(A_n)_{n \in \N}$ iz $\F$, ki je nenaraščajoča glede na inkluzijo: $A_n \supset A_{n+1} ~\forall n \in N$.

\item[(vi)] Za vsak $A \in \F$ je
$$\F \big|_A ~:=~ \set{B \cap A \mid B \in \F}$$
$\sigma$-algebra na $A$ in $\mu_A := \mu\big|_{\F\big|_A}$ je mera na $(A, \F\big|_A)$. 

\end{enumerate}

\end{trditev}
\vspace{0.5cm}

\begin{definicija}

$\mu_A := \mu\big|_{\F\big|_A}$ rečemo \textit{mera $\mu$ zožana na $A$} oz. \textit{zožitev $\mu$ \hbox{na $A$}}.

\end{definicija}
\vspace{0.5cm}

% *************************************************************************************************

\subsection{Merljive preslikave in generirane $\sigma$-algebre}
\vspace{0.5cm}

\begin{definicija}

Naj bo $\A \subset 2^\Omega$:
$$\sigma_\Omega(\A) ~:=~ \bigcap\set{\F \in 2^{2^\Omega} \mid \F ~\sigma\text{-algebra na}~\Omega~\text{in}~\F\supset\A},$$
rečemo $\sigma$-algebra z $A$ na $\Omega$. Če sta $\B_1$ in $\B_2$ obe $\sigma$-algebri na $\Omega$, potem definiramo
$$\B_1 \vee \B_2 ~:=~ \sigma_\Omega(\B_1 \cup \B_2)$$
in ji rečemo skupek $\B_1$ in $\B_2$. Bolj splošno, če imamo družino $\sigma$-algebr $(B_\lambda)_{\lambda\in\Lambda}$ na $\Omega$, potem postavimo
$$\bigvee_{\lambda\in\Lambda} \B_\lambda ~:=~ \sigma_\Omega\oklepaj{\bigcup_{\lambda\in\Lambda} \B_\lambda}.$$

\end{definicija}
\vspace{0.5cm}

\begin{definicija}

Naj bo $f: \Omega \rightarrow \Omega'$. Če je dana $\sigma$-algebra $\F'$ na $\Omega'$, potem definiramo
$$\sigma^{\F'}(f) ~:=~ \set{f^{-1}(A) \mid A \in \F'}.$$
Začetno strukturo $f$ glede na $\F'$ (tudi, $\sigma$-algebra generirana s $f$ glede na $\F'$). Če je dana $\sigma$-algebra $\F$ na $\Omega$, potem definiramo
$$\sigma_\F^{\Omega'}(f) ~:=~ \set{A' \in 2^{\Omega'} \mid f^{-1}(A') \in \F}$$
končno strukturo $f$ na $\Omega'$ glede na $\F$. Če sta dani $\sigma$-algebri $\F$ na $\Omega$ in $\sigma$-algebra $\F'$ na $\Omega$, potem rečemo: $f$ je $\F/\F'$-merljiva $\diff$
$$f^{-1}(A') \in \F, ~~~\forall A' \in \F.$$

\end{definicija}
\vspace{0.5cm}

\begin{definicija}

Če je $\F$ $\sigma$-algebra na $\Omega$ in je $\F'$ $\sigma$-algebra na $\Omega'$, potem označimo
$$\F/\F' ~:=~ \set{f \in \Omega'^\Omega \mid f ~\text{je}~ \F/\F'\text{-merljiva}}.$$

\end{definicija}
\vspace{0.5cm}

\begin{definicija}

Za $A \subset \Omega$ definiramo $\1_{A_\Omega}: \Omega \rightarrow \set{0,1}$,
$$\1_{A_\Omega}(x) ~:=~ \begin{cases}
1\,; ~&x \in A, \\
0\,; ~&x \notin A,
\end{cases}, ~~~x \in \Omega,$$
ki ji rečemo \textit{indikatorska funkcija $A$ na ambientnem prostoru $\Omega$}.\footnote{Ponavadi namesto $\1_{A_\Omega}$ pišemo le $\1_A$.}

\end{definicija}
\vspace{0.5cm}

\begin{trditev}

Za $\sigma$-algebre $\F,\G,\H$, kjer $f \in \F/\G$ in $g \in \G/\H$ je
$$g \circ f ~\in~ \F/\H.$$

\end{trditev}
\vspace{0.5cm}

\begin{trditev}

Naj bo $f: \Omega \rightarrow \Omega'$:
\begin{enumerate}

\item[(i)] Za $\sigma$-algebro $\F'$ na $\Omega'$ je $\sigma^{\F'}(f)$ $\sigma$-algebra na $\Omega$; ona je najmanjša (glede na inkluzijo) $\sigma$-algebra $\G$ na $\Omega$, da je $f \in \G/\F'$.

\item[(ii)] Za $\sigma$-algebro $\F$ na $\Omega$ je $\sigma_F^{\Omega'}(f)$ $\sigma$-algebra na $\Omega'$; ona je največja (glede na inkluzijo) $\sigma$-algebra $\G'$ na $\Omega$, da je $f \in \F/\G$.

\item[(iii)] Za $\sigma$-algebro $\F$ na $\Omega$ in $\sigma$-algebro $\F'$ na $\Omega'$ je
$$f \in \F/\F' ~\iff~ \sigma^{\F'}(f) \subset \F ~\iff~ \F' \subset \sigma_\F^{\Omega'}(f).$$

\item[(iv)] Naj bo $\A' \sigma 2^{\Omega'}$ ter $\F$ $\sigma$-algebra na $\Omega$. Potem je
$$f \in \F/\sigma_{\Omega'}(\A') ~\iff~ (f^{-1}(A') \in \F, ~\forall A' \in \A').$$
Velja tudi 
$$\sigma^{\sigma_{\Omega'}(\A')}(f) ~=~ \sigma_\Omega(\set{f^{-1}(A') \mid A' \in \A'}).$$

\end{enumerate}

\end{trditev}
\vspace{0.5cm}

\begin{definicija}

\textit{Sled $\A$ na $A$} definiramo kot
$$\A \big|_A ~:=~ \set{B \cap A \mid B \in \A}.\footnote{\text{Zapis je isti kot za zožitev, vendar ne pomeni isto.}}$$

\end{definicija}
\vspace{0.5cm}

\begin{trditev}[Sledi komutirajo v generirani $\sigma$-algebri]

Naj bo $\A \subset 2^\Omega$ in $A \subset \Omega$. Potem je
$$\sigma_A(\A \big|_A) ~=~ \sigma_\Omega(\A) \big|_A.$$

\end{trditev}
\vspace{0.5cm}

% *************************************************************************************************

\pagebreak

% #################################################################################################

\end{document}