\documentclass[11pt]{article}
\usepackage[utf8]{inputenc}
\usepackage[slovene]{babel}

\usepackage{amsthm}
\usepackage{amsmath, amssymb, amsfonts}
\usepackage{relsize}
\usepackage{mathrsfs}
\usepackage{bbm}
\usepackage{xcolor}
\usepackage{mathabx} % ta \bigtimes

\newcommand{\R}{\mathbb{R}}
\newcommand{\N}{\mathbb{N}}
\renewcommand{\P}{\mathbb{P}}
\newcommand{\E}{\mathbb{E}}
\newcommand{\A}{\mathcal{A}}
\newcommand{\BB}{\mathcal{B}}
\newcommand{\CC}{\mathcal{C}}
\newcommand{\F}{\mathcal{F}}
\newcommand{\G}{\mathcal{G}}
\renewcommand{\H}{\mathcal{H}}
\newcommand{\I}{\mathcal{I}}
\newcommand{\Z}{\mathcal{Z}}
\newcommand{\X}{\mathcal{X}}
\renewcommand{\O}{\mathcal{O}}
\newcommand{\D}{\mathcal{D}}
\newcommand{\M}{\mathcal{M}}
\renewcommand{\L}{L}
\newcommand{\B}{\mathscr{B}}
\newcommand{\C}{\mathcal{C}}
\newcommand{\LL}{\mathscr{L}}
\newcommand{\EE}{\mathcal{E}}
\renewcommand{\c}{\mathsf{c}}
\newcommand{\diff}{\overset{\text{def}}{\iff}}
\newcommand{\imp}{~\Rightarrow~}
\newcommand{\set}[1]{\{#1\}}
\newcommand{\oklepaj}[1]{\left(#1\right)}
\newcommand{\1}{\mathbbm{1}}
\newcommand{\e}{\mathbbm{e}}
\newcommand{\f}{\mathbbm{f}}
\newcommand{\rr}{[-\infty,\infty]}
\newcommand{\floor}[1]{\lfloor#1\rfloor}
\newcommand{\id}{\text{id}}
\newcommand{\supp}{\text{supp}}
\newcommand{\pr}{\text{pr}}
\renewcommand{\Re}{\mathfrak{Re}}
\renewcommand{\Im}{\mathfrak{Im}}

\theoremstyle{definition}
\newtheorem{definicija}{Definicija}[section]

\theoremstyle{definition}
\newtheorem{trditev}{Trditev}[section]

\theoremstyle{definition}
\newtheorem{izrek}{Izrek}[section]

\theoremstyle{definition}
\newtheorem{metoda}{Metoda}[section]

\newtheorem*{posledica}{Posledica}
\newtheorem*{opomba}{Opomba}
\newtheorem*{komentar}{Komentar}
\newtheorem{lema}{Lema}
\newtheorem*{dokaz}{Dokaz}
\newtheorem*{posplošitev}{Posplošitev}
\newtheorem*{dogovor}{Dogovor}
\newtheorem*{sklep}{Sklep}

\title{Verjetnost z mero - definicije, trditve in izreki}
\author{Oskar Vavtar \\
po predavanjih profesorja Matija Vidmarja}
\date{2021/22}

\begin{document}
\maketitle
\pagebreak
\tableofcontents
\pagebreak

% #################################################################################################

\section{Merljivost in mera}
\vspace{0.5cm}

% *************************************************************************************************

\subsection{Merljive množice}
\vspace{0.5cm}

\begin{definicija}

Naj bo $\A \subset 2^\Omega$ (t.j. $\A \in 2^{2^\Omega}$). Potem rečemo, da je $\A$ \textit{zaprta} za:
\begin{itemize}

\item $\c^\Omega$ (t.j. za komplement v $\Omega$) 
$$\diff ~~~\forall A: (A \in \Omega \imp \Omega \setminus A \in \A);$$

\item $\cap$ (t.j. za preseke)
$$\diff ~~~A_1 \cap A_2 \in \A ~~\text{brž ko je}~ \set{A_1,A_2} \subset A;$$

\item $\cup$ (t.j. za unije)
$$\diff ~~~A_1 \cup A_2 \in \A ~~\text{brž ko je}~ \set{A_1,A_2} \subset A;$$

\item $\setminus$ (t.j. za razlike)
$$\diff ~~~A_1 \setminus A_2 \in \A ~~\text{brž ko je}~ \set{A_1,A_2} \subset A;$$

\item $\sigma\cap$ (t.j. za števne preseke)
$$\diff ~~~\bigcap_{n\in\N} A_n \in \A ~~\text{za vsako zaporedje}~ (A_n)_{n\in\N} ~\text{iz}~ \A;$$

\item $\sigma\cup$ (t.j. za števne unije)
$$\diff ~~~\bigcup_{n\in\N} A_n \in \A ~~\text{za vsako zaporedje}~ (A_n)_{n\in\N} ~\text{iz}~ \A.$$

\end{itemize}

\end{definicija}
\vspace{0.5cm}

\begin{definicija}

$\A$ je \textit{$\sigma$-algebra} na $\Omega$
\begin{align*}
&\diff ~~~(\Omega,\A) ~~\text{je merljiv prostor} \\
&\diff ~~~\emptyset \in \A ~\text{in}~ \A ~\text{je zaprt za}~ \c^\Omega ~\text{in za}~ \sigma\cup.
\end{align*}
V primeru, da $\A$ je $\sigma$-algebra na $\Omega$:
\begin{itemize}
	\item $A$ je $\A$-merljiva $~\diff~ A \in \A;$
	\item $\B$ je pod-$\sigma$-algebra $~\diff~ \B$ je $\sigma$-algebra na $\Omega$ in $\B \subset \A.$
\end{itemize}

\end{definicija}
\vspace{0.5cm}

\begin{trditev}

Naj bo $\A \subset 2^\Omega$ zaprta za $\c^\Omega$ in naj bo $\emptyset \in \A$. Potem je $\A$ $\sigma$-algebra na $\Omega$, če je $\A$ zaprta za števne preseke, in v tem primeru je $\A$ zaprta za $\cap$, $\cup$ in $\setminus$.

\end{trditev}
\vspace{0.5cm}

% *************************************************************************************************

\subsection{Mere}
\vspace{0.5cm}

\begin{definicija}

Naj bo $(\Omega,\F)$ merljiv prostor in $\mu: \F \rightarrow [0,\infty]$. $\mu$ je \textit{mera} na $(\Omega,\F)$ $\diff$
\begin{itemize}
	\item $\mu(\emptyset) ~=~ 0;$
	\item $\mu\mathlarger{\oklepaj{\bigcup_{n\in\N}} ~=~ \sum_{n\in\N} \mu(A_n)}$ za vsako zaporedje $(A_n)_{n\in\N}$ iz $\F$, ki sestoji iz paroma disjunktnih dogodkov.
\end{itemize}
Lastnosti:
\begin{itemize}
	\item Mera $\mu$ na $(\Omega,\F)$ je \textit{končna} $\diff$ $\mu(\Omega)<\infty$.
	\item Mera $\mu$ na $(\Omega,\F)$ je \textit{verjetnostna}\footnote{Tudi: $\mu$ je \textit{verjetnost}.} $\diff$ $\mu(\Omega) = 1$
	
	\item Mera $\mu$ na $(\Omega,\F)$ je \textit{$\sigma$-končna} $\diff$ obstaja zaporedje $(A_n)_{n\in\N}$ v $\F$, da je
	\begin{align*}
	\bigcup_{n\in\N} ~&=~ \Omega ~~~\text{in} \\
	\mu(A_n) ~&<~ \infty, ~~~\forall n \in \N
	\end{align*}
	
\end{itemize}
$(\Omega,\F,\mu)$ je prostor z mero $\diff$ $\mu$ je mera na $(\Omega,\F)$. Če je $\mu$ mera na $(\Omega,\F)$ potem je $\mu(\Omega)$ \textit{masa} mere $\mu$. Če je $A \in \F$, potem je:
\begin{itemize}
	\item $A$ je \textit{$\mu$-zanemarljiv} $\diff$ $\mu(A) = 0$;
	\item $A$ je \textit{$\mu$-trivialna} $\diff$ $A$ ali $\Omega\setminus A$ je $\mu$-zanemarljiva
\end{itemize}
Če imamo poleg tega še lastnost $P(\omega)$ v $\omega \in A$, potem
\begin{itemize}
	\item $P(\omega)$ drži \textit{$\mu$-skoraj povsod} ($\mu$-s.p.) v $\omega \in A$ $\diff$ 
	$$A_{\neg P} ~:=~ \set{\omega \in \Omega \mid \neg P(\omega) \in \F ~\text{in}~ \mu(A_{\neg P}) = 0};$$
	\item $P(\omega)$ drži \textit{$\mu$-skoraj gotovo} ($\mu$-s.g.) $\diff$ $P(\omega)$ drži $\mu$-skoraj povsod in $\mu$ je verjetnostna.
\end{itemize}
$P$ drži $\mu$-skoraj povsod na $A$ $\diff$ $P(\omega)$ drži $\mu$-skoraj povsod v $\omega \in A$. Podobno za ostale.

\end{definicija}
\vspace{0.5cm}

\begin{trditev}

Naj bo $\mu$ mera na $(\Omega, \F)$. Potem:
\begin{enumerate}

\item[(i)] $\mu$ je aditivna: 
$$\mu(A \cup B) ~=~ \mu(A) + \mu(B)$$
za vsaki disjunktivni množici $A,B \in \F$.

\item[(ii)] $\mu$ je monotona:
$$\mu(A) ~\leq~ \mu(B),$$
če je $A \subset B$ in $A,B \in \F$

\item[(iii)] $\mu$ je zvezna od spodaj:
$$\mu\oklepaj{\bigcup_{n \in \N} A_n} ~=~ {\uparrow\text{-}\lim_{n \rightarrow \infty}} \mu(A_n)$$
za vsako zaporedje $(A_n)_{n \in \N}$ iz $\F$, ki je nepadajoče glede na inkluzijo: $A_n \subset A_{n+1} ~\forall n \in \N$.

\item[(iv)] $\mu$ je števno subaditivna:
$$\mu\oklepaj{\bigcup_{n \in \N} A_n} ~\leq~ \sum_{n \in \N} \mu(A_n)$$
za vsako zaporedje $(A_n)_{n \in \N}$ iz $\F$.

\item[(v)] Naj bo $\mu$ končna:
$$\mu(\Omega \setminus A) ~=~ \mu(\Omega) - \mu(A) ~\forall A \in \F.$$
Naprej, $\mu$ je zvezna od zgoraj:
$$\mu\oklepaj{\bigcap_{n \in \N} A_n} ~=~ {\downarrow\text{-}\lim_{n \rightarrow \infty}}\mu(A_n)$$
za vsako zaporedje $(A_n)_{n \in \N}$ iz $\F$, ki je nenaraščajoča glede na inkluzijo: $A_n \supset A_{n+1} ~\forall n \in N$.

\item[(vi)] Za vsak $A \in \F$ je
$$\F \big|_A ~:=~ \set{B \cap A \mid B \in \F}$$
$\sigma$-algebra na $A$ in $\mu_A := \mu\big|_{\F\big|_A}$ je mera na $(A, \F\big|_A)$. 

\end{enumerate}

\end{trditev}
\vspace{0.5cm}

\begin{definicija}

$\mu_A := \mu\big|_{\F\big|_A}$ rečemo \textit{mera $\mu$ zožana na $A$} oz. \textit{zožitev $\mu$ \hbox{na $A$}}.

\end{definicija}
\vspace{0.5cm}

% *************************************************************************************************

\subsection{Merljive preslikave in generirane $\sigma$-algebre}
\vspace{0.5cm}

\begin{definicija}

Naj bo $\A \subset 2^\Omega$:
$$\sigma_\Omega(\A) ~:=~ \bigcap\set{\F \in 2^{2^\Omega} \mid \F ~\sigma\text{-algebra na}~\Omega~\text{in}~\F\supset\A},$$
rečemo $\sigma$-algebra z $A$ na $\Omega$. Če sta $\B_1$ in $\B_2$ obe $\sigma$-algebri na $\Omega$, potem definiramo
$$\B_1 \vee \B_2 ~:=~ \sigma_\Omega(\B_1 \cup \B_2)$$
in ji rečemo skupek $\B_1$ in $\B_2$. Bolj splošno, če imamo družino $\sigma$-algebr $(B_\lambda)_{\lambda\in\Lambda}$ na $\Omega$, potem postavimo
$$\bigvee_{\lambda\in\Lambda} \B_\lambda ~:=~ \sigma_\Omega\oklepaj{\bigcup_{\lambda\in\Lambda} \B_\lambda}.$$

\end{definicija}
\vspace{0.5cm}

\begin{definicija}

Naj bo $f: \Omega \rightarrow \Omega'$. Če je dana $\sigma$-algebra $\F'$ na $\Omega'$, potem definiramo
$$\sigma^{\F'}(f) ~:=~ \set{f^{-1}(A'); ~A' \in \F'}.$$
Začetno strukturo $f$ glede na $\F'$ (tudi, $\sigma$-algebra generirana s $f$ glede na $\F'$). Če je dana $\sigma$-algebra $\F$ na $\Omega$, potem definiramo
$$\sigma_\F^{\Omega'}(f) ~:=~ \set{A' \in 2^{\Omega'} \mid f^{-1}(A') \in \F}$$
končno strukturo $f$ na $\Omega'$ glede na $\F$. Če sta dani $\sigma$-algebri $\F$ na $\Omega$ in $\sigma$-algebra $\F'$ na $\Omega$, potem rečemo: $f$ je $\F/\F'$-merljiva $\diff$
$$f^{-1}(A') \in \F, ~~~\forall A' \in \F'.$$

\end{definicija}
\vspace{0.5cm}

\begin{definicija}

Če je $\F$ $\sigma$-algebra na $\Omega$ in je $\F'$ $\sigma$-algebra na $\Omega'$, potem označimo
$$\F/\F' ~:=~ \set{f \in \Omega'^\Omega \mid f ~\text{je}~ \F/\F'\text{-merljiva}}.$$

\end{definicija}
\vspace{0.5cm}

\begin{definicija}

Za $A \subset \Omega$ definiramo $\1_{A_\Omega}: \Omega \rightarrow \set{0,1}$,
$$\1_{A_\Omega}(x) ~:=~ \begin{cases}
1\,; ~&x \in A, \\
0\,; ~&x \notin A,
\end{cases}, ~~~x \in \Omega,$$
ki ji rečemo \textit{indikatorska funkcija $A$ na ambientnem prostoru $\Omega$}.\footnote{Ponavadi namesto $\1_{A_\Omega}$ pišemo le $\1_A$.}

\end{definicija}
\vspace{0.5cm}

\begin{trditev}

Za $\sigma$-algebre $\F,\G,\H$, kjer $f \in \F/\G$ in $g \in \G/\H$ je
$$g \circ f ~\in~ \F/\H.$$

\end{trditev}
\vspace{0.5cm}

\begin{trditev}

Naj bo $f: \Omega \rightarrow \Omega'$:
\begin{enumerate}

\item[(i)] Za $\sigma$-algebro $\F'$ na $\Omega'$ je $\sigma^{\F'}(f)$ $\sigma$-algebra na $\Omega$; ona je najmanjša (glede na inkluzijo) $\sigma$-algebra $\G$ na $\Omega$, da je $f \in \G/\F'$.

\item[(ii)] Za $\sigma$-algebro $\F$ na $\Omega$ je $\sigma_F^{\Omega'}(f)$ $\sigma$-algebra na $\Omega'$; ona je največja (glede na inkluzijo) $\sigma$-algebra $\G'$ na $\Omega$, da je $f \in \F/\G$.

\item[(iii)] Za $\sigma$-algebro $\F$ na $\Omega$ in $\sigma$-algebro $\F'$ na $\Omega'$ je
$$f \in \F/\F' ~\iff~ \sigma^{\F'}(f) \subset \F ~\iff~ \F' \subset \sigma_\F^{\Omega'}(f).$$

\item[(iv)] Naj bo $\A' \sigma 2^{\Omega'}$ ter $\F$ $\sigma$-algebra na $\Omega$. Potem je
$$f \in \F/\sigma_{\Omega'}(\A') ~\iff~ (f^{-1}(A') \in \F, ~\forall A' \in \A').$$
Velja tudi 
$$\sigma^{\sigma_{\Omega'}(\A')}(f) ~=~ \sigma_\Omega(\set{f^{-1}(A') \mid A' \in \A'}).$$

\end{enumerate}

\end{trditev}
\vspace{0.5cm}

\begin{definicija}

\textit{Sled $\A$ na $A$} definiramo kot
$$\A \big|_A ~:=~ \set{B \cap A \mid B \in \A}.\footnote{\text{Zapis je isti kot za zožitev, vendar ne pomeni isto.}}$$

\end{definicija}
\vspace{0.5cm}

\begin{trditev}[Sledi komutirajo v generirani $\sigma$-algebri]

Naj bo $\A \subset 2^\Omega$ in $A \subset \Omega$. Potem je
$$\sigma_A(\A \big|_A) ~=~ \sigma_\Omega(\A) \big|_A.$$

\end{trditev}
\vspace{0.5cm}

\begin{trditev}

Naj bo $f: \Omega \rightarrow \Omega'$ in naj bo $\F$ $\sigma$-algebra na $\Omega$ ter $\F'$ $\sigma$-algebra na $\Omega'$.
\begin{enumerate}

\item[(i)] Če je $A' \subset \Omega'$ in $f: \Omega \rightarrow A'$, potem je 
$$f \in \F/\F' ~\iff~ f \in \F/(\F' \big|_{A'}).$$

\item[(ii)] Če je $A \in \Omega$ in $f \in \F/\F'$, potem
$$f\big|_A \in (\F\big|_A)/\F'.$$

\item[(iii)] Če je $(A_i)_{i \in \N}$ zaporedje v $\F$ in $\Omega = \bigcup_{i \in \N} A_i$ in je $f\big|_{A_i} \in (\F\big|_{A_i})/\F' \\\forall i \in \N$, potem je
$$f \in \F/\F'$$.

\end{enumerate}

\end{trditev}
\vspace{0.5cm}

% *************************************************************************************************

\subsection{Borelove množice na razširjeni realni osi $[-\infty,\infty]$ \\in Borelova merljivost numeričnih funkcij}
\vspace{0.5cm}

\begin{definicija}

Definirajmo \textit{razširjeno realno os}:
\begin{align*}
\rr ~&:=~ \R \cup \set{\infty} \cup \set{-\infty} \\
[-\infty, a] ~&:=~ \set{-\infty} \cup (-\infty,a] ~~~\text{za}~ a \in \R \cup \set{-\infty}
\end{align*}
Relacijo $\leq$ na $\R$ razširimo na $[-\infty,\infty]$ kot sledi:
$$-\infty \leq x \leq \infty ~~~\forall x \in \rr.$$
Temu ustrezno imamo ``$(<) ~:=~ (\leq) \setminus (=)$'', itd.

\end{definicija}
\vspace{0.5cm}

\begin{definicija}

\textit{Borelovo $\sigma$-algebro} na $\rr$ definiramo kot
$$\B_{\rr} ~:=~ \sigma_{\rr}(\set{[-\infty,a] \mid a \in \R}).$$
Za $A \subset \rr$ je
$$\B_A ~:=~ \B_{\rr} \big|_A$$
Borelova $\sigma$-algebra na $A$. Elementom Borelovih $\sigma$-algebr pravimo \textit{Borelove množice}. 

\end{definicija}
\vspace{0.5cm}

\begin{definicija}

Funkcija $f$ je \textit{numerična}, če je $\Z_f \in \rr$.

\end{definicija}
\vspace{0.5cm}

\begin{definicija}

Če je funkcija $f$ numerična:
\begin{itemize}

\item $\sigma(f) ~:=~ \sigma^{\B_{\rr}}(f)$;

\item če je $\F$ $\sigma$-algebra na domeni $f$, je $f$ $\F$-merljiva $\diff$ $f$ je $\F/\B_{\rr}$-merljiva;

\item če je $g: \D_f \rightarrow \rr$, je
\begin{align*}
g \wedge f ~&:=~ \min\set{g,f}\footnote{$g \wedge f: \D_f \rightarrow \rr$ in $(g \wedge f)(\omega) := \min\set{g(\omega),f(\omega)} ~\forall \omega \in \D_f$} \\
g \vee f ~&:=~ \max\set{g,f}.
\end{align*}
Definiramo \textit{pozitivni} in \textit{negativni del $f$}:
\begin{align*}
f^+ ~&:=~ f \vee 0 \\
f^- ~&:=~ (-f) \vee 0
\end{align*}

\end{itemize}

\end{definicija}
\vspace{0.5cm}

\begin{opomba}
~
\begin{itemize}
	\item $f ~=~ f^+ - f^-$
	\item $|f| ~=~ f^+ + f^-$
\end{itemize}

\end{opomba}
\vspace{0.5cm}

\begin{definicija}

Dogovorimo se
\begin{align*}
0 \cdot (\pm \infty) ~&:=~ 0 ~=:~ (\pm \infty) \cdot 0 \\
\infty + (-\infty) ~&:=~ 0 ~=:~ (-\infty) + \infty.
\end{align*}
Preostanek aritmetike na $\rr$ definiramo na naraven način, npr.
\begin{align*}
a \cdot \infty ~&:=~ \text{sgn}(a) \cdot \infty ~~~\text{za}~ a \in \rr \setminus \set{0} \\
a + \infty ~&:=~ \infty ~~~\text{za}~ a \in (-\infty, \infty] \\
\infty - \infty ~&:=~ \infty + (-\infty) ~=~ 0 \\
&itd.
\end{align*}

\end{definicija}
\vspace{0.5cm}

\begin{trditev}

Če je $A \subset \rr$ in je $f: A \rightarrow \rr$ zvezna, potem je $f \in \B_A/\B_{\rr}$. Če je $\set{f,g} \subset \F/\B_{\rr}$ za $\sigma$-algebro $\F$, potem je 
$$\set{f+g,f \cdot g} \subset \F/\B_{\rr}$$ 
in 
$$\set{\set{f \leq g}, \set{f=g}, \set{f<g}} \subset \F$$.

\end{trditev}
\vspace{0.5cm}

\begin{trditev}

Naj bo $\F$ $\sigma$-algebra in $(f_n)_{n \in \N}$ zaporedje v $\F/\B_{\rr}$. Potem je
$$\set{\sup_{n \in \N} f_n, \inf_{n \in \N} f_n, \limsup_{n \rightarrow \infty} f_n, \liminf_{n \rightarrow \infty} f_n} ~\subset~ \F/\B_{\rr}.$$
Če je $f_n \geq 0 ~\forall n \in \N$, potem je 
$$\sum_{n \in \N} f_n ~\in~ \F/\B_{[0,\infty]}.$$ 

\end{trditev}
\vspace{0.5cm}

\begin{definicija}

Naj bo $\F$ $\sigma$-algebra. Za $\set{f,g} \subset \F/\B_{\rr}$ je
$$\set{f \vee g, f \wedge g, f^+, f^-, |f|} ~\subset~ \F/\B_{\rr}.$$
Za zaporedje $(f_n)_{n \in \N}$ v $\F/\B_{\rr}$ je
$$\set{\set{f_n~\text{konverg., ko}~n\rightarrow\infty},\set{f_n~\text{konverg. v}~\R,~\text{ko}~n\rightarrow\infty},\set{\lim_{n\rightarrow\infty} f_n = f_\infty}} ~\subset~ \F.$$

\end{definicija}
\vspace{0.5cm}

% *************************************************************************************************

\subsection{Argumenti monotonega razreda}
\vspace{0.5cm}

\begin{definicija}

Naj bo $\F$ $\sigma$-algebra na $\Omega$ in $f: \Omega \rightarrow [0, \infty)$:
$$f ~\text{je}~ \F\text{-enostavna} ~\diff~ f \in \F/\B_{[0, \infty)} ~\text{in}~ \Z_f ~\text{je končna}.$$

\end{definicija}
\vspace{0.5cm}

\begin{trditev}

Naj bo $(\Omega, \F)$ merljiv prostor in $f: \Omega \rightarrow [0, \infty]$. Potem je $f$ $\F$-enostavna $\iff$
$$f ~=~ \sum_{i=1}^n c_i \1_{A_i},$$
za neke $c_i$, $i \in [n]$, iz $[0, \infty)$, neke $A_i$, $i \in [n]$, iz $\F$ in nek $n \in \N$. Naprej; če je $f \in \F/\B_{[0, \infty]}$, potem je 
$$\oklepaj{(2^{-n}\floor{2^n f}) \wedge n}_{n \in \N}$$
zaporedje $\F$-enostavnih funkcij, ki ne padajo k $f$ (celo enakomerno na vsaki množici na kateri je $f$ omejena).

\end{trditev}
\vspace{0.5cm}

\begin{posledica}[Izrek o monotonem razredu]

Naj bo $\F$ $\sigma$-algebra na $\Omega$ in \hbox{$\M \subset \F/\B_{[0,\infty]}$}. Če je
$$\1_A \in \M \quad \forall A \in \F$$
in je $\M$ zaprta za nenegativne linearne kombinacije (je stožec)\footnote{Pomeni: $$\set{m_1,m_2}\subset\M, ~\set{c_1,c_2}\subset(0,\infty) ~\Rightarrow~ c_1 m_1 + c_2 m_2 \in \M$$} in je $\M$ zaprta za nepadajoče limite\footnote{Pomeni: $(f_n)_{n \in \N}$ nepadajoče zaporedje iz $\M$, potem je $$\lim_{n \rightarrow \infty} f_n \in \M$$} potem je
$$\M ~=~ \F/\B_{[0, \infty]}.$$

\end{posledica}
\vspace{0.5cm}

\begin{trditev}[Doob-Dynkinova faktorizacijska lema]

Naj bo $X: \Omega \rightarrow A$, $(A, \A)$ merljiv prostor. Potem je
$$Y \in \sigma^\A(X)/\B_{\rr} ~\iff~ \exists h \in \A/\B_{\rr}, ~\text{da je}~ Y = h \circ X = h(X).$$

\end{trditev}
\vspace{0.5cm}

\begin{definicija}

Naj bo $\D \subset 2^\Omega$. $D$ je Dynkinov sistem (tudi $\lambda$-sistem) na $\Omega$ $\diff$
\begin{itemize}
	\item $\Omega \in \D$,
	\item $B\setminus A \in \D $ brž ko je $ \D \ni A \subset B \in \D$,
	\item če je $(A_i)_{i \in \N}$ je nepadajoče zaporedje v $\D$ je tudi $\bigcup_{n \in \N} A_n \in \D$.
\end{itemize}
$\D$ je $\pi$-sistem $\diff$ $\D$ je zaprt za $\cap$.

\end{definicija}
\vspace{0.5cm}

\begin{trditev}

Naj bo $\D \subset 2^\Omega$. Potem je $\D$ Dynkinov sistem $\iff$
\begin{itemize}
	\item $\Omega \in \D$,
	\item $\D$ zaprta za $\c^\Omega$,
	\item $(A_i)_{i \in \N}$ zaporedje iz $\D$, $A_i \cap A_j = \emptyset$ za $i \neq j$ iz $\N$ $\Longrightarrow$ $\bigcup_{n \in \N} A_i \in \D$.
\end{itemize}
$\D$ je $\sigma$-algebra na $\Omega$ $\iff$ $\D$ je $\lambda$-sistem na $\Omega$ in $\pi$-sistem.

\end{trditev}
\vspace{0.5cm}

\begin{definicija}

Za $\L \subset 2^\Omega$ postavimo
$$\lambda_\Omega(\L) ~:=~ \bigcap\set{\D \in 2^{2^\Omega} \mid \D ~\text{je}~\lambda\text{-sistem in}~\D\supset\L}.$$

\end{definicija}
\vspace{0.5cm}

\begin{trditev}

Naj bo $\L$ $\pi$-sistem in $\L \subset 2^\Omega$. Potem je
$$\lambda_\Omega(\L) ~=~ \sigma_\Omega(\L).$$

\end{trditev}
\vspace{0.5cm}

\begin{posledica}[$\pi$-$\lambda$ izrek/Dynkinova lema]

Naj bo $\L$ $\pi$-sistem in $\D$ $\lambda$-sistem na $\Omega$, $\L \subset \D$. Potem je
$$\sigma_\Omega(\L) ~\subset~ \D.$$

\end{posledica}
\vspace{0.5cm}

\begin{trditev}

Naj bosta $\mu, \nu$ meri na merljivem prostoru $(E,\EE)$, $\L \subset \EE$ $\pi$-sistem, $\sigma_E(\L) = \EE$. Predpostavimo, da je $\mu |_\L = \nu |_\L$ in da obstaja zaporedje $(L_n)_{n \in \N}$ iz $\L$, ki je nepadajoče ali sestoji iz paroma disjunktnih množic, in za katerega je
\begin{itemize}
	\item $\mu(L_n) ~=~ \nu(L_n) ~<~ \infty$,
	\item $\bigcup_{n \in \N} L_n ~=~ E$.
\end{itemize}
Potem je 
$$\mu ~=~ \nu$$.

\end{trditev}
\vspace{0.5cm}

% *************************************************************************************************

\subsection{Lebesgue-Stieltjesova mera}
\vspace{0.5cm}

\begin{izrek}[Lebesgue-Stieltjesov izrek]

Naj bo $F: \R \rightarrow \R$, nepadajoča in zvezna z desne (\textit{ca'd}). Potem obstaja natanko ena mera $\mu$ na $\B_\R$, da je
$$\mu([a,b]) ~=~ F(b) - F(a) \quad \forall a\leq b \in \R.$$

\end{izrek}
\vspace{0.5cm}

\begin{definicija}

$\mu$ iz prejšnjega izreka rečemo \textit{mera prirejena $F$ v Lebesgue-Stieltjesovem smislu} in jo označimo z $dF$. V posebne primernu primeru, ko je $F = \id_\R$ ji rečemo \textit{Lebesgueva mera} in jo označimo
$$\LL ~:=~ d(\id_\R).$$

\end{definicija}
\vspace{0.5cm}

\begin{trditev}

Naj bo $F: \R \rightarrow \R$ \textit{ca'd}, nepadajoča. Potem je $dF$:
\begin{itemize}
	\item $\sigma$-končna $\iff$ je $F$ omejena:
	$$dF(\R) ~=~ \lim_{n \rightarrow \infty} dF((-n,n])$$
	\item verjetnostna $\iff$ $\lim_\infty F - \lim_{-\infty} F = 1$.
\end{itemize}
Za $x \in \R$ je 
$$dF(\set{x}) ~=~ F(x) - F(x^-),$$
$\set{x} = \bigcap_{n \in \N} (x - \frac{1}{n}, x]$.

\end{trditev}
\vspace{0.5cm}

% *************************************************************************************************

\pagebreak

% #################################################################################################

\section{Integracija na merljivih prostorih}
\vspace{0.5cm}

% *************************************************************************************************

\subsection{Lebesgueov integral}
\vspace{0.5cm}

\begin{definicija}

Naj bo $(\Omega, \F, \mu)$ prostor z mero $f \in \F/\B_{\rr}$.
\begin{enumerate}

\item[(a)] Za $f$, ki je $\F$-enostavna postavimo
$$\int f\,d\mu ~:=~ \sum_{a \in \Z_f} d\mu(\set{f=a}) ~=~ \sum_{a \in \Z_f} d\mu(f^{-1}(\set{a})).$$

\item[(b)] Za $f \geq 0$, ki ni $\F$-enostavna postavimo
$$\int f\,d\mu ~:=~ \sup\set{\int g\,d\mu \mid g \leq f,~ g ~\F\text{-enostavna}}.$$

\item[(c)] Za $\neg(f \geq 0)$, ki ni $\F$-enostavna postavimo
$$\int f\,d\mu ~:=~ \int f^+\,d\mu - \int f^-\,d\mu.$$

\end{enumerate}

\end{definicija}
\vspace{0.5cm}

\begin{dogovor}

$$\mu[f] ~=~ \mu^x[f(x)] ~:=~ \int f(x)\,\mu(dx) ~:=~ \int f\,d\mu$$
Če je še $A \in \F$, potem označimo še
$$\mu[f; A] ~:=~ \mu^x[f(x); x \in A] ~:=~ \int_A f(x)\,\mu(dx) ~:=~ \int_A f\,d\mu ~:=~ \int f\1_A\,d\mu.$$
Integral $f$ proti $\mu$ je \textit{dobro definiran} $\diff$
$$\int f^+\,d\mu ~\wedge \int f^-\,d\mu ~<~ \infty;$$
$f$ je $\mu$-integrabilna $\diff$
$$\int f^+\,d\mu ~\vee \int f^-\,d\mu ~<~ \infty.$$

\end{dogovor}
\vspace{0.5cm}

\begin{definicija}

Naj bo $(\Omega, \F, \mu)$ prostor z mero:
$$\L^1(\mu) ~:=~ \set{f \in \F/\B \mid f ~\text{je}~ \mu\text{-integrabilna}}.$$
Za $g: \Omega \rightarrow \mathbb{C}$ z $\set{\mathfrak{R}(g), \mathfrak{I}(g)} \subset \L^1(\mu)$ je
$$\int g\,d\mu ~:=~ \int \mathfrak{R}(g)\,d\mu + i\int \mathfrak{I}(g)\,d\mu.$$

\end{definicija}
\vspace{0.5cm}

\begin{izrek}

Naj bo $(\Omega, \F, \mu)$ prostor z mero. Integral ima naslednje lastnosti:
\begin{enumerate}

\item[(i)] Aditivnost:
$$\int f+g \,d\mu  ~=~ \int f \,d\mu + \int g \,d\mu,$$
za $\set{f,g} \subset \F/\B_{\rr}$ z $\mu[f^-] \vee \mu[g^-] < \infty$

\item[(ii)] Integral indikatorja:
$$\int \1_A \,d\mu ~=~ \mu(A), \quad \forall A \in \F.$$
V posebnem primeru je $\mu[0] = 0$ in torej $\mu[f^+] - \mu[f^-] = \mu[f] ~\forall f \in \F/\B_{\rr}$.

\item[(iii)] Integrali, ki so $0$ in so končni:
\\za $f \in \F/\B_{[0,\infty]}$:
\begin{itemize}
	\item $\mu[f] = 0$ $\iff$ $f=0$ s.p.-$\mu$  
	\item $\mu[f] < \infty$ $\Longrightarrow$ $f<\infty$ s.p.-$\mu$.
\end{itemize}

\item[(iv)] Trikotniška neenakost:
$$\left|\int f \,d\mu\right| \leq \int |f| \,d\mu, \quad \forall f \in \F/\B_{\rr}.$$

\item[(v)] Integral ``ne vidi'' množic z mero $0$:
\\če je $\set{f,g} \subset \F/\B_{\rr}$ in je $f=g$ s.p.-$\mu$, potem je $\mu[f]=\mu[g]$ in je $\mu[f]$ d.d. $\iff$ $\mu[g]$ je d.d.

\item[(vi)] Monotonost:
če je $\set{f,g} \subset \F/\B_{\rr},$ $g\leq f$ in $\mu[g^-] < \infty$, potem je
$$\int g\,d\mu ~\leq~ \int f \,d\mu.$$

\item[(vii)] Homogenost:
$$\int c f \,d\mu ~=~ c \int f \,d\mu$$
za vse $f \in \F/\B_{\rr}$ za katere je $\mu[cf^-] \wedge \mu[cf^+] < \infty$, za \hbox{$\forall c \in \rr$}.

\end{enumerate}
Vsi integrali v (i),(ii),(iii),(vi) so d.d. Enako velja za (vii), razen, ko je $c=0$ in $\mu[f^+]=\mu[f^-]=\infty$.

\end{izrek}
\vspace{0.5cm}

\begin{trditev}

Naj bosta $a \leq b$ realni števili in $f: [a,b] \rightarrow \R$. Če je $f$ zvezna, potem je $\LL$-integrabilna in 
$$\int_{[a,b]} f \,d\LL ~=~ \int_a^b f(x)\,dx.$$

\end{trditev}
\vspace{0.5cm}

% *************************************************************************************************

\subsection{Konvergenčni izreki s posledicami}
\vspace{0.5cm}

\begin{izrek}

Naj bo $(\Omega, \F,\mu)$ prostor z mero in $(f_n)_{n \in \N}$ zaporedje iz $\F/\B_{\rr}$.
\begin{itemize}

\item[(i)] Naj bo $g \in \F/\B_{[0,\infty]}$ z $\mu[g] < \infty$ in $f_n^- \leq g ~\forall n \in \N$. Potem velja:
\begin{itemize}
	\item[(a)] Polzveznost od spodaj (Fatou):
	$$\int \liminf_{n \rightarrow \infty} f_n \,d\mu ~\leq \liminf_{n \rightarrow \infty} \int f_n \,d\mu$$
	\item[(b)] Monotona konvergenca (L\'evy):
	$$\int \lim_{n \rightarrow \infty} f_n \,d\mu ~=~ \uparrow\text{-}\lim_{n \rightarrow \infty} \int f_n d\mu$$
\end{itemize}

\item[(ii)] Naj bo $g \in \F/\B_{[0,\infty]}$ $\mu$-integrabilna z $|f_n| \leq g ~\forall n \in \N$. Potem velja dominirana konvergenca (Lebesgue):
$$\lim_{n \rightarrow \infty} \int |f_n - \lim_{m \rightarrow \infty} f_m| \,d\mu ~=~ 0$$ 
in v posebnem je
$$lim_{n \rightarrow \infty} \int f_n \,d\mu ~=~ \int \lim_{n \rightarrow \infty} f_n \,d\mu,$$
če seveda $\exists \lim_{m \rightarrow \infty} f_m$ (povsod).

\end{itemize}

\end{izrek}
\vspace{0.5cm}

\begin{posledica}

Naj bo $(\Omega, \F, \mu)$ prostor z mero in $(f_n)_{n \in \N}$ zaporedje iz $\F/\B_{[0,\infty]}$. Potem je
$$\int \sum_{n \in \N} f_n \,d\mu ~=~ \sum_{n \in \N} \int f_n \,d\mu,$$
kjer so integrali d.d.

\end{posledica}
\vspace{0.5cm}

\begin{posledica}

Naj bo $(\mu_n)_{n \in \N}$ zaporedje mer na merljivem prostoru $(\Omega, \F)$. Potem je $\sum_{n \in \N} \mu_n$ mera na $(\Omega, \F):$
$$\oklepaj{\sum_{n \in \N} \mu_n}(A) ~=~ \sum_{n \in \N} \mu_n(A), \quad A \in \F.$$
Poleg tega je za $\forall f \in \F/\B_{\rr}$:
$$\int f \,d(\sum_{n \in \N} \mu_n) ~\text{je d.d.} ~\iff~ \oklepaj{\sum_{n \in \N} \int f^+ \,d\mu_n} \wedge \oklepaj{\sum_{n \in \N} \int f^- \,d\mu_n} < \infty$$
in tedaj je
$$\int f \,d(\sum_{n \in \N} \mu_n) ~=~ \sum_{n \in \N} \int f \,d\mu_n.$$

\end{posledica}
\vspace{0.5cm}

\begin{definicija}

Naj bo $(\Omega, \F, \mu)$ porostor z mero, $(\Omega', \F')$ merljiv prostor, $f \in \F/\F'$. Potem definiramo
$$f *_{\F'} \mu \quad \text{oz.} \quad \mu \circ_{\F'} f^{-1} \quad \text{oz.} \quad \mu_{f_{\F'}}$$
kot preslikavo $f *_{\F'} \mu: \F' \rightarrow [0,\infty]$, dano s predpisom
$$(f *_{\F'} \mu)(A') ~:=~ \mu(f^{-1}(A')), \quad A' \in \F'.$$
To preslikavo imenujemo \textit{potisk mere $\mu$ naprej pod $f$ glede na $\F'$}. Če je $\mu$ verjetnostna, rečemo temu \textit{porazdelitev}.

\end{definicija}
\vspace{0.5cm}

\begin{posledica}[Izrek o sliki mer]

Naj bo $(\Omega, \F, \mu)$ prostor z mero, $(\Omega', \F')$ merljiv prostor, $f \in \F/\F'$. Potem je $f * \mu$ mera na $\F'$, verjetnostna, če je $\mu$ verjetnostna. Če je $g \in \F'/\B_{\rr},$ je
$$\int g \,d(f*\mu) ~=~ \int g \circ f \,d\mu,$$
pri čemer je integral na levi d.d. $\iff$ je to res za integral na desni.

\end{posledica}
\vspace{0.5cm}

\begin{posledica}[Odvajanje pod integralskim znakom]

Naj bo $(\X, \Sigma, \mu)$ prostor z mero, $\O$ odprt v $\R$. $F: \X \times \O \rightarrow \R$ in naj velja:
\begin{itemize}
	\item $\forall t \in \O$ je $\F(\cdot,t) \in \L^1(\mu)$;
	\item $\forall x \in \X$ je $\F(x,\cdot)$ odvedljiva.
\end{itemize}
Naj naprej $\exists g \in \Sigma/\B_{[0,\infty]}$ z $\mu[g]<\infty$ tako, da je
$$\left|\frac{\partial F}{\partial t}(x,t)\right| ~\leq~ g(x), \quad \forall (x,t) \in \X \times \O.$$
Potem velja:
\begin{enumerate}
	\item[(a)] $\forall t \in \O$ je $\oklepaj{\X \ni x \mapsto \frac{\partial F}{\partial t}(x,t)} \in \L^1(\mu)$;
	\item[(b)] $\oklepaj{\O \ni t \mapsto \int F(x,t) \,\mu(dx)}$ je odvedljiva;
	\item[(c)] $t \in \O$:
	$$\frac{d}{dt} \int F(x,t) \,\mu(dx) ~=~ \int \frac{\partial F}{\partial t}(x,t)\,\mu(dx).$$
\end{enumerate}

\end{posledica}
\vspace{0.5cm}

% *************************************************************************************************

\subsection{Rezultati, ki se tičejo menjave vrsrnega reda \\integracije}
\vspace{0.5cm}

\begin{definicija}

Naj bosta $(\Omega, \F)$ in $(\Omega', \F')$ merljiva prostora. Potem definiramo
$$\F \otimes \F' ~:=~ \sigma_{\Omega\times\Omega'} \oklepaj{\set{A \times A' \mid (A,A') \in \F\times\F'}},$$
in ji rečemo \textit{produktna $\sigma$-algebra $\F$ in $\F'$}.

\end{definicija}
\vspace{0.5cm}

\begin{trditev}

Če je $A \subset \R^2$ in $f: A \rightarrow \rr$ zvezna, potem  je 
$$f \in \B_A/\B_{\rr}.$$

\end{trditev}
\vspace{0.5cm}

\begin{trditev}

Naj bosta $(\Omega, \F)$ in $(\Omega', \F')$ merljiva prostora. Potem je $\F \otimes \F'$ najmanjša (glede na inkluzijo) $\sigma$-algebra na $\Omega \times \Omega'$ glede na katero sta merljivi kanonični projekciji, tj. $\F \otimes \F'$ je najmanjša $\sigma$-algebra $\G$ na $\Omega\times\Omega'$, da je:
\begin{itemize}
	\item $\oklepaj{\Omega \times \Omega' \ni (\omega, \omega') \mapsto \omega} \in \G/\F$;
	\item $\oklepaj{\Omega \times \Omega' \ni (\omega, \omega') \mapsto \omega'} \in \G/\F'$.
\end{itemize}
Naprej, če je $f \in \F \otimes \F'/\B_{\rr}$, potem je $f(\omega, \cdot) \in \F'/\B_{\rr},$ $\forall \omega \in \Omega$ in $f(\cdot, \omega') \in \F/\B_{\rr},$ $\forall \omega' \in \Omega'$. Obratno, naj bo $(G,\G)$ merljiv prostor; potem je
$$(f,f') \in \G/\F\otimes\F' ~\iff~ f \in \G/\F ~\text{in}~ f' \in \G/\F'.$$
 
\end{trditev}
\vspace{0.5cm}

\begin{izrek}

Naj bosta $(\Omega, \F, \mu)$ in $(\Omega', \F', \mu')$ prostora z mero, $\mu$ in $\mu'$ \hbox{$\sigma$-končni}.
\begin{enumerate}

\item[(i)] Obstaja natanko ena mera $\nu$ na $\F\otimes\F'$, ki jo označimo $\mu\times\mu'$, za katero velja
$$\nu(A\times A') ~=~ \mu(A)\mu'(A'), \quad \forall (A,A')\in\F\times\F'.$$

\item[(ii)] Naj bo $f \in (\F\otimes\F')/\B_{\rr}$ in naj velja
\begin{enumerate}
	\item[(a)] $f \geq 0$ \textit{(Tonelli)} ali
	\item[(b)] $\int |f|\,d(\mu\times\mu') < \infty$ \textit{(Fubini)} ali
	\item[(c)] $\iint f^-(\omega,\omega')\,\mu(d\omega)\,\mu'(d\omega') ~\wedge~ \iint f^-(\omega,\omega')\,\mu'(d\omega')\,\mu(d\omega) < \infty$
\end{enumerate}

\end{enumerate}
Potem je 
\begin{itemize}
	\item $\oklepaj{\Omega'\ni\omega' \mapsto \int f(\omega,\omega')\,\mu(d\omega)} \in \F'/\B_{\rr}$;
	\item $\oklepaj{\Omega\ni\omega \mapsto \int f(\omega,\omega')\,\mu'(d\omega')} \in \F/\B_{\rr}$;
	\item $\int f^-(\omega,\omega')\,\mu(d\omega) < \infty$ s.p.-$\mu'$ v $\omega'$;
	\item $\int f^-(\omega,\omega')\,\mu'(d\omega') < \infty$ s.p.-$\mu$ v $\omega$;
\end{itemize}
in 
\begin{align*}
\int f\,d(\mu\times\mu') ~&=~ \iint f(\omega,\omega')\,\mu(d\omega)\,\mu'(d\omega')  \\
&=~ \iint f(\omega,\omega')\,\mu'(d\omega')\,\mu(d\omega).
\end{align*}
Vsi zunanji integrali zgoraj so d.d.

\end{izrek}
\vspace{0.5cm}

\begin{definicija}

Notacijo $\mu\times\mu'$ zadržimo, $\mu\times\mu'$ rečemo \textit{produkt $\mu$ in $\mu'$}. 

\end{definicija}
\vspace{0.5cm}

\begin{trditev}

$(\Omega,\F,\mu)$ prostor z mero, $(\Omega',\F')$ merljiv prostor, $X \in \F/\F'$, $(A,\A)$ še en merljiv prostor, da je
$$D_A ~:=~ \set{(x,x) \mid x \in A} \in \A\otimes\A$$
in $\set{f,g} \subset \F'/\A$. Potem je $f(X) = g(X)$ s.p.-$\mu$ $\iff$ $f=g$ s.p.-$X_*\mu$.

\end{trditev}
\vspace{0.5cm}

% *************************************************************************************************

\subsection{Nedoločena integracija in absolutna zveznost}
\vspace{0.5cm}

\begin{definicija}

Naj bosta $(\Omega,\F,\mu)$ prostor z mero, $f \in \F'/\B_{\rr}$ in naj bo integral $f$ pod $\mu$ d.d. Potem preslikavi
$$f \cdot \mu ~:=~ \oklepaj{\F \in A \mapsto \int_A f\,d\mu}$$
rečemo \textit{nedoločeni integral $f$ proti $\mu$}\footnote{Beri: glede na $\mu$.}, ali tudi tudi \textit{$\mu$-nedoločeni integral $f$}.

\end{definicija}
\vspace{0.5cm}

\begin{definicija}

Naj bosta $\mu$ in $\nu$ dve meri na merljivem prostoru $(\Omega,\F)$. $\mu$ je absolutno zvezna glede na $\nu$ (pišemo $\mu \ll \nu$) $\diff$
$$\nu(A) = 0 ~\Rightarrow~ \mu(A), \quad \forall A \in \F.$$
$\mu$ je ekvivalentna $\nu$, (pišemo $\mu\sim\nu$) $\diff$
$$\mu \ll \nu \quad \text{in} \quad \nu \ll \mu.$$

\end{definicija}
\vspace{0.5cm}

\begin{trditev}

Naj bo $(\Omega,\F,\mu)$ prostor z mero in $f \in \F/\B_{[0,\infty]}$. Potem je $f \cdot \mu$ mera, ki je absolutno zvezna glede na $\mu$; naprej
$$\int g \,d(f\cdot\mu) ~=~ \int g\,\color{teal}f(d\mu)\color{black}$$
za vse $g \in \F/\B_{\rr},$ pri čemer je integral na levi d.d. $\iff$ je integral na desni d.d. in v slednjem primeru je 
$$g \cdot (f \cdot \mu) ~=~ (gf) \cdot \mu.$$
Če je $f>0$ s.p.-$\mu$, potem je $f\cdot\mu \sim \mu$.

\end{trditev}
\vspace{0.5cm}

% *************************************************************************************************

\begin{trditev}

Naj bo $(X, \A, \mu)$ prostor z mero, $\set{f,g} \subset \A/\B_{\rr}$.
\begin{enumerate}

\item[(a)] Denimo, da je $\int_{\set{f>g}} f^+\,d\mu \vee \int_{\set{f>g}} g^-\,d\mu <\infty$ in \hbox{$\int_{\set{f>g}} f\,d\mu \leq \int_{\set{f>g}} g\,d\mu$}. Potem je 
$$f~<~g \quad \text{s.p.-}\mu.$$

\item[(b)] Denimo, da je $\mu$ $\sigma$-končna, \hbox{$\oklepaj{\int f^+\,d\mu \wedge \int f^-\,d\mu} \vee \oklepaj{\int g^+\,d\mu \wedge \int g^-\,d\mu} < \infty$} in $\int_A f\,d\mu \leq \int_A g\,d\mu$, $\forall A \in \A$. Potem je
$$f ~\leq~ g \quad \text{s.p.-}\mu.$$ 

\end{enumerate}

\end{trditev}
\vspace{0.5cm}

\begin{posledica}

Naj bo $(X,\A,\mu)$ prostor z mero, $\set{f,g} \subset \A/\B_{\rr}$. Denimo, da je $\int_A f\,d\mu = \int_A g\,d\mu$, $A \in \A$, pri čemer sta $\mu[f]$ in $\mu[g]$ d.d. Če je $f$ (ali/torej $g$) $\mu$-integrabilna ali če je $\mu$ $\sigma$-končna, potem je
$$f~=~g \quad \text{s.p.-}\mu.$$ 
V primeru, ko sta $f$ in $g$ $\mu$-integrabilna, potem je, ceteris paribus, enakost $\int_A f\,d\mu = \int_A g\,d\mu$ dovolj preveriti za $A \in \Pi \cup \set{X}$, kjer je $\Pi$ nek $\pi$-sistem, ki generira $\A$ na $X$.

\end{posledica}
\vspace{0.5cm}

\begin{izrek}[Radon-Nikodym]

Naj bosta $\mu$ in $\nu$ $\sigma$-končni meri na istem merljivem prostoru $(\Omega, \F)$, $\mu \ll \nu$. Potem obstaja $f \in \F/\B_{[0,\infty]}$, enolična do enakosti s.p.-$\mu$, za katero je
$$\mu ~=~ f\cdot\nu,$$
$f>0$ s.p.-$\mu$.

\end{izrek}
\vspace{0.5cm}

\begin{definicija}

Funkcijo $f$ iz zgornjega izreka označimo z $$\frac{d\mu}{d\nu}.$$ Rečemo ji \textit{Radon-Nikodymov odvod}.

\end{definicija}
\vspace{0.5cm}

\begin{posledica}

Naj bodo $\mu \ll \nu \ll \lambda$ $\sigma$-končne mere na $\sigma-algebri$. Potem je $\mu \ll \lambda$ in 
$$\frac{d\mu}{d\lambda} ~=~ \frac{d\mu}{d\nu}\cdot\frac{d\nu}{d\lambda} \quad \text{s.p.-}\lambda.$$
Torej, če je $\mu \sim \nu$,
$$1 ~=~ \frac{d\mu}{d\nu}\frac{d\nu}{d\mu} \quad \text{s.p.-}\mu ~\text{in}~ \text{s.p.-}\nu.$$

\end{posledica}
\vspace{0.5cm}

% *************************************************************************************************

\subsection{Prostori $L$ in nekaj integralskih neenakosti}
\vspace{0.5cm}

\begin{definicija}

Naj bo $(\Omega, \F, \mu)$ prostor z mero, $p \in [1,\infty)$ in $f \in \F/\B_{\rr}$, definiramo:
\begin{itemize}
\item prostor $L^p$:
\begin{align*}
\|f\|_{p_\mu} ~&:=~ \oklepaj{\int |f|^p\,d\mu}^\frac{1}{p}, \\
L^p(\mu) ~&:=~ \set{f \in \F/\B_\R;~\|f\|_{p_\mu}<\infty},
\end{align*}
\item prostor $L^\infty$:
\begin{align*}
\|f\|_{\infty_\mu} ~&:=~ \inf\set{M \in [0,\infty];~|f| \leq M ~\text{s.p.-}\mu}, \\
L^\infty(\mu) ~&:=~ \set{f \in \F/\B_\R;~\|f\|_{\infty_\mu}<\infty}.
\end{align*}
\end{itemize}
Za zaporedje $(f_n)_{n \in \N_0}$ v $L^q(\mu)$, $q \in [1,\infty]$, rečemo da $f_n \xrightarrow{n \rightarrow \infty} f_0$ v $L^q(\mu)$ $\diff$ 
$$\|f_n-f_0\|_{q_\mu} ~\xrightarrow{n \rightarrow \infty}~ 0.$$
Za $\set{f,g} \subset L^2(\mu)$,
$$\langle f,g \rangle = \int fg\,d\mu.$$

\end{definicija}
\vspace{0.5cm}

\begin{trditev}

Naj bo $\mu$ končna mera in $p \leq q$, $\set{p,q} \subset [1,\infty]$. Potem je
$$L^q(\mu) ~\subset~ L^p(\mu).$$

\end{trditev}
\vspace{0.5cm}

\begin{trditev}

Naj bo $(\Omega, \F, \mu)$ prostor z mero, $\set{f,g} \subset \F/\B_{\rr}$. Imamo sledeče neenakosti:
\begin{enumerate}

\item[(i)] Markov:
$$\mu[f;f\geq a] ~\geq~ a \cdot \mu(f \geq a), \quad \forall a \in \rr;$$
torej $\mu[f] \geq a\mu(f \geq a)$ za $\forall a \in [0,\infty]$, brž ko je $f \geq 0$.

\item[(ii)] Minkowski:
$$\|f+g\|_p ~\leq~ \|f\|_p + \|g\|_p, \quad \forall p \in [1,\infty]$$

\item[(iii)] H\"older:
$$\|fg\|_1 ~\leq~ \|f\|_p \|g\|_q, \quad \forall \set{p,q} \subset [1,\infty],~p^{-1} + q^{-1}=1.$$
V posebnem $p=q=2$, Cauchy-Schwartzova neenakost.

\item[(iv)] Jensen: \\
naj bo $\mu$ verjetnostna, $f \in L^1(\mu)$, $\varphi: I \rightarrow \R$ konveksna, $I$ odprt interval, $f: \Omega \rightarrow I$. Potem je $\varphi \in \B_I/\B_\R$, $\int\oklepaj{\varphi \circ f}^-\,d\mu < \infty$, $\int f\,d\mu \in I$ in 
$$\int \varphi \circ f\,d\mu ~\geq~ \varphi\oklepaj{\int f\,d\mu}.$$ 
Najprej, za $\forall p \in [1,\infty]$ je $\|\cdot\|_p$ seminorma na $L^p(\mu)$, ki je realni linearen prostor, in v njem je $\|\cdot\|_p$-limita zaporedje, če obstaja, s.p.-$\mu$ enolično določena; obstaja \textit{čee} je dano zaporedje Cauchyjevo v seminormi $\|\cdot\|_p$. Končno, $\langle \cdot,\cdot \rangle$ je skalarni semiprodukt na $L^2(\mu)$.

\end{enumerate}

\end{trditev}
\vspace{0.5cm}

\pagebreak

% #################################################################################################

\section{Verjetnost kot normalizirana mera}
\vspace{0.5cm}

% *************************************************************************************************

\subsection{Osnovni pojmi}
\vspace{0.5cm}

\begin{definicija}

\textit{Verjetnostni prostor} je prostor z mero $(\Omega, \F, \P)$ pri čemer je $\P$ verjetnostna. Naj bo $(\Omega, \F, \P)$ verjetnostni prostor; $A$ je $\P$-skoraj gotov ($\P$-s.g.) $\diff$ $A \in \F$ in $\P(A) = 1$. \\

Če je $(E, \EE)$ merljiv prostor, potem elementom $\F/\EE$ rečemo \textit{slučajni elementi} z vrednostmi v $(E,\EE)$; v posebnem primeru, ko je $(E,\EE) = (\R, \B_\R)$ jim rečemo \textit{slučajne spremenljivke}. \\

Za slučajni element $X$: $X \sim_\P Q$ $\diff$ $X$ ima zakon $Q$ pod $\P$, t.j. $X_*\P = Q$. Za dva slučajna elementa, ki imata vrednosti v istem merljivem prostoru rečemo, da sta \textit{enako porazdeljena} $\diff$ imata isti zakon. \textit{Porazdelitvena funkcija} slučajne spremenljivke $X$ je preslikava $F_X: \R \rightarrow [0,1]$ dana z $F_X(x) = \P(X \leq x)$ za $x \in \R$. \\

Slučajna spremenljivka $X$ je \textit{diskretna} $\diff$ $\exists C$ števna podmnožica $\R$, da je $\P(X \in C) = 1$. Slučajna spremenljivka $X$ je \textit{absolutno zvezna} $\diff$ $\P_X \ll \LL$. Slučajna spremenljivka $X$ je \textit{zvezna} $\diff$ $F_X$ je zvezna. \\

\textit{Bivarianten slučajni vektor} je element $\F/\B_{\R^2}$, torej slučajen vektor z vrednostmi v $(\R^2, \B_{\R^2})$; $(X,Y)$ je absolutno zvezen $\diff$ $\P_{(X,Y)} \ll \LL^2$, itd.

\end{definicija}
\vspace{0.5cm}

\begin{trditev}

Naj bo $X$ slučajni element na verjetnostnem prostoru $(\Omega, \F, \P)$ z vrednostmi v merljivem prostoru $(E, \EE)$ in $f \in \EE/\B_{\rr}$. Potem je
$$\P[f(X)] ~=~ \P_X[f],$$
pri čemer je upanje na levi strani d.d. \textit{čee} je d.d upanje na desni strani.\\

Za slučajno spremenljivko $X$ je $F_X$ ca'd, $\uparrow$ in $\lim_{-\infty} F_X=0$, \hbox{$\lim_\infty F_X=1$.} \\

Če je $X$ diskretna slučajna spremenljivka, potem obstaja najmanjša števna množica $C \subset \R$, da $\P(X \in C) = 1$, ki ji rečemo podpora $X$, označimo s $\supp(X)$:
$$\supp(X) ~=~ \set{x \in \R;~\P(X=x)>0},$$
narprej, za $f \in \B_\R/\B_{\rr}$ je
$$\P[f(X)] ~=~ \sum_{x \in \supp(X)} f(x)\P(X=x),$$
če je le $\sum_{x \in \supp(X)} f^+(x)\P(X=x) \wedge \sum_{x \in \supp(X)} f^-(x)\P(X=x) < \infty$.\footnote{Potem je tudi $\P[f(X)]$ d.d.} \\

Če je $X$ absolutno zvezna, potem je zvezna in obstaja do $\LL$-s.p. natančno enolična funkcija $f \in \B_\R/\B_{[0,\infty]}$ za katero je $\P_X = f \cdot \LL$; ta $f$ označimo $f_X$ in ji rečemo gostota $X$; naprej za $g \in \B_\R/\B_{\rr}$ je
$$\P[g(X)] ~=~ \int g(x)f_X(x)\,\LL(dx),$$
pri čemer so integrali d.d. brž ko je $\int g^+ f\,d\LL \wedge \int g^- f\,d\LL < \infty$. \\

Končno, za to da je slučajna spremenljivka $X$ absolutno zvezna je posebno in zadostno, da $\exists f \in \B_\R/\B_{[0,\infty]}$, da je
$$\P(X \leq x) ~=~ \int_{[-\infty,x]} f\,d\LL, \quad \forall x \in \R$$
in v tem primeru je $f$ gostota za $X$.\footnote{Bolj splošno je ekvivalentno preveriti $\P(X \in A) = \int_A f\,d\LL$ za $A \in \Pi \cup \set{\R}$, kjer je $\Pi$ nek $\pi$-sistem, ki generira $\B_\R$ na $\R$.}\,\footnote{$\P_X = dF_X$}

\end{trditev}
\vspace{0.5cm}

\begin{definicija}

Zadržimo notacijo za gostoto $f_X$, $\supp(X)$; za diskretno slučajno spremenljivko $X$. Definiramo \textit{verjetnostno masno funkcijo} $X$ kot
$$p_X ~:=~ \oklepaj{\supp(X) \ni x \mapsto \P(X=x)}.$$

\end{definicija}
\vspace{0.5cm}

\begin{trditev}
~
\begin{enumerate}

\item[(1)] Naj bo $(\Omega, \F, \P)$ verjetnostni prostor. Če je $X$ slučajna spremenljivka, potem je $F_X \in \B_\R/\B_{[0,1]}$ in $\F_X(x) \sim_\P \LL_{[0,1]}$ \textit{čee} je $X$ zvezna.

\item[(2)] Obratno, naj bo $U \sim_\P \LL_{[0,1]}$. Če je $F: \R \rightarrow \R$ porazdelitvena funkcija (ca'd, $\uparrow$, $\lim_{-\infty} F=0$, $\lim_\infty F = 1$) in če vpeljemo
$$F^{\leftarrow}(x) ~:=~ \inf\set{v \in \R;~F(v)>u}, \quad x \in (0,1),$$
potem je
$$F^{\leftarrow}(U) ~\sim_\P~ dF.\footnote{$F^{\leftarrow}$ je \textit{desni inverz} $F$ oz. \textit{kvantilna funkcija} $F$.}$$

\end{enumerate}

\end{trditev}
\vspace{0.5cm}

\begin{definicija}

Naj bo $(\Omega, \F, \P)$ verjetnostni prostor, $(X_n)_{n \in \N}$ zaporedje v $\F/\B_\R$ in $X \in \F/\B_\R$. $(X_n)_{n \in \N}$ \textit{konvergira k $X$ v $\P$-verjetnosti} $\diff$ 
$$\forall \varepsilon \in (0,\infty): ~\P(|X_n - X| \geq \varepsilon) \xrightarrow{n \rightarrow \infty} 0.$$

\end{definicija}
\vspace{0.5cm}

\begin{trditev}

Naj bo $(\Omega, \F, \P)$ verjetnostni prostor. Če je $(X_n)_{n \in \N}$ zaporedje v $\F/\B_\R$, ki konvergira k $X \in \F/\B_\R$ s.g.-$\P$ ali v $L^q(\P)$ za nek $q \in [1,\infty]$, potem konvergira tudi v $\P$-verjetnosti.

\end{trditev}
\vspace{0.5cm}

% *************************************************************************************************

\subsection{Neodvisnost}
\vspace{0.5cm}

\begin{definicija}

Naj bo $(\Omega, \F, \P)$ verjetnostni prostor. Za družino $\C=(\C_\lambda)_{\lambda\in\Lambda}$ podmnožic $\F$ rečemo, da je neodvisnost (pod $\P$) $\diff$ za vsako končno neprazno $I \subset \Lambda$, $\forall C_\lambda \in \C_\lambda$, $\lambda \in I$
$$\P\oklepaj{\bigcap_{\lambda \in I} C_\lambda} ~=~ \prod_{\lambda \in I} \P(C_\lambda).$$\\

Za neodvisni podmnožici $\BB$ in $\CC$ $\sigma$-algebre $\F$ je $\BB$ neodvisna od $\CC$ (pod $\P$) $\diff$ $(\BB,\CC)$ je neodvisnost (pod $\P$). Za dogodka $B$ in $C$ iz $\F$ je $B$ neodvisen od $C$ $\diff$ $\set{B}$ je neodvisna od $\set{C}$. \\

Za slučajni element $Z$ z vrednostmi v merljivem prostoru $(E,\EE)$ in za $\BB \subset \F$ je $\BB$ neodvisna od $Z$ (pod $\P$ glede na $\EE$) $\diff$ $\BB$ je neodvisna od $\sigma^\EE(Z)$.\footnote{Nasploh, neodvisnost slučajnih elementov pomeni neodvisnost $\sigma$-algebr, ki so generirane z njimi.}

\end{definicija}
\vspace{0.5cm}

\begin{trditev}

Naj bo $(\Omega, \F, \P)$ verjetnostni prostor, $X$ slučajni element z vrednostmi v $(E, \EE)$, $Y$ slučajni element t vrednostmi v $(A, \A)$. Potem je $(X,Y) \in \F/\EE\otimes\A$. Najprej, $X$ in $Y$ sta neodvisna od $\P$ \textit{čee} \hbox{$P_{(X,Y)} ~=~ \P_X \times \P_Y$};
v tem primeru je za $f \in (\EE \otimes \A)/\B_{\rr}$
$$\P[f(X,Y)] ~=~ \P_{(X,Y)}[f] ~=~ \int \P[f(x,Y)]\P_X(dx),$$
če je le $\P[f^-(X,Y)] \wedge \P[f^+(X,Y)] < \infty$; v posebnem, za $g \in \EE/\B_{\rr}$, $h \in \A/\B_{\rr}$ je
$$\P[g(X)h(Y)] ~=~ \P[g(X)]\P[h(Y)].$$

\end{trditev}
\vspace{0.5cm}

\begin{trditev}

Naj bo $(\Omega, \F, \P)$ verjetnostni prostor in $(\C_\lambda)_{\lambda \in \Lambda}$ družina $\pi$-sistemov, $\C_\lambda \subset \F ~\forall \lambda \in \Lambda$. Če je $(\C_\lambda)_{\lambda \in \Lambda}$ neodvisnost, potem je tudi $(\sigma_\Omega(\C_\lambda))_{\lambda\in\Lambda}$ neodvisnost.

\end{trditev}
\vspace{0.5cm}

\begin{trditev}

Naj bo $(\Omega, \F, \P)$ verjetnostni prostor in $(X,Y)$ absolutno zvezen bivariantni slučajni vektor. Označimo
$$f_{(X,Y)} ~:=~ \frac{d\P_{(X,Y)}}{d\LL^2}.$$
Potem sta $X$ in $Y$ absolutno zvezni slučjni spremenljivki;
\begin{align*}
f_X(x) ~&=~ \int f_{(X,Y)}(x,y)\,\LL(dy) \quad \text{s.p.-}\LL ~\text{v}~x, \\
f_Y(y) ~&=~ \int f_{(X,Y)}(x,y)\,\LL(dx) \quad \text{s.p.-}\LL ~\text{v}~y
\end{align*}
ter sta $X$ in $Y$ neodvisni \textit{čee}
$$f_{(X,Y)} ~=~ f_X(x)f_Y(y) \quad \text{s.p.-}\LL^2 ~\text{v}~ (x,y).$$
Če sta $X$ in $Y$ neodvisni, potem je
$$\P[g(X,Y)] ~=~ \iint g(x,y) f_X(x) f_Y(y) \,\LL(dx)\,\LL(dy)$$
za $\forall g \in \B_{\R^2}/\B_{\rr}$ z $\P[g^+(X,Y)] \wedge \P[g^-(X,Y)] < \infty$.

\end{trditev}
\vspace{0.5cm}

\begin{definicija}

Ohranimo notacijo $f_{(X,Y)}$ za gostoto slučajnega vektorja $(X,Y)$.

\end{definicija}
\vspace{0.5cm}

\begin{definicija}

Naj bo $\oklepaj{(\Omega_\lambda, \F_\lambda)}_{\lambda\in\Lambda}$ družina merljivih prostorov. Definiramo
$$\bigotimes_{\lambda\in\Lambda} \F_\lambda ~:=~ \bigvee_{\lambda \in \Lambda} \sigma^{\F_\lambda}(\pr_\lambda),$$
kjer so 
\begin{align*}
\pr_\lambda: \prod_{\mu \in \Lambda} \Omega_\mu &\rightarrow \Omega_\lambda, \quad \lambda \in \Lambda \\
(\omega_\mu)_{\mu \in \Lambda} &\mapsto \omega_\lambda
\end{align*}
kanonične projekcije. Za $\sigma$-algebro $\F$ in množico $\Lambda$ je 
$$\F^{\otimes\Lambda} ~:=~ \bigotimes_{\lambda\in\Lambda} \F.$$

\end{definicija}
\vspace{0.5cm}

\begin{trditev}\label{ref:3.7}

Naj bo $\oklepaj{(\Omega, \F_\lambda)}_{\lambda\in\Lambda}$ družina merljivih prostorov.
\begin{itemize}

\item[(i)] $\bigotimes_{\lambda\in\Lambda} \F_X$ je najmanjša $\sigma$-algebra na $\prod_{\lambda\in\Lambda} \Omega_\lambda$ glede na katero so merljive vse kanonične projekcije $\pr_\lambda$, $\lambda\in\Lambda$.

\item[(ii)] Za merljiv prostr $(\Omega, \F)$ in družino funkcij $f_\lambda: \Omega \rightarrow \Omega_\lambda$, $\lambda\in\Lambda$, je
$$(f_\lambda)_{\lambda\in\Lambda} \in \F/(\bigotimes_{\lambda\in\Lambda} \F_\lambda) ~\iff~ f_\lambda \in \F/\F_\lambda, ~ \forall \lambda \in \Lambda.$$

\item[(iii)] Naj bo $\mu_\lambda$ verjetnost na $(\Omega_\lambda, \F_\lambda)$, $\forall \lambda \in \Lambda$. Potem obstaja na $\oklepaj{\prod_{\lambda\in\Lambda}\Omega_\lambda, \bigotimes_{\lambda\in\Lambda} \F_\lambda}$ natanko ena verjetnost $\mu$ za katero so $(\pr_\lambda)_{\lambda\in\Lambda}$ neodvisni pod $\mu$ in
$$(\pr_\lambda)_*\mu ~=~ \mu_\lambda.$$

\end{itemize}

\end{trditev}
\vspace{0.5cm}

\begin{trditev}

Naj bo $(\Omega, \F, \P)$ verjetnostni prostor in naj bo $f_\lambda$ slučajni element z vrednostmi v $(E_\lambda, \EE_\lambda)$, $\lambda\in\Lambda$. Potem je $(f_\lambda)_{\lambda\in\Lambda}$ neodvisnost pod $\P$ \textit{čee}
$$\oklepaj{(f_\lambda)_{\lambda\in\Lambda}}_{*_{\otimes_{\lambda\in\Lambda}\EE_\lambda}}\P ~=~ \bigtimes_{\lambda\in\Lambda} ({f_{\lambda*}} \P).$$

\end{trditev}
\vspace{0.5cm}

\begin{definicija}

$\mu$ iz točke (iii) v trditvi \ref{ref:3.7} označimo z 
$$\bigtimes_{\lambda\in\Lambda} \mu_\lambda,$$
\textit{produkt verjetnostnih mer} $\mu_\lambda$, $\lambda\in\Lambda$. Za verjetnost $\mu$ in množico $\Lambda$ velja
$$\mu^{\times\Lambda} ~:=~ \bigtimes_{\lambda\in\Lambda} \mu_\lambda.$$

\end{definicija}
\vspace{0.5cm}

% *************************************************************************************************

\subsection{Pogojevanje}
\vspace{0.5cm}

\begin{trditev}

Če je $(\Omega, \A, \P)$ verjetnostni prostor in $A \in \A$ z $\P(A)>0$, potem je $(A, \A\big|_A, \frac{1}{\P(A)}\P_A)$ verjetnostni prostor in
$$\oklepaj{\frac{1}{\P(A)}\P}[f] ~=~ \frac{\P(f;A)}{\P(A)}, \quad \forall f \in \oklepaj{\A\big|_A}/\B_{\rr}.$$

\end{trditev}
\vspace{0.5cm}

\begin{definicija}

Naj bo $(\Omega, \A, \P)$ verjetnostni prostor in $A \in \A$ z $\P(A)>0$. Za $B \in \A$ definiramo
$$\P(B \mid A) ~:=~ \frac{\P(B \cap A)}{\P(A)},$$
\textit{verjetnost $B$ pogojno na $A$ pod $\P$}. Pišemo
$$\P(\cdot \mid A) ~=~ \frac{1}{\P(A)}\P_A$$
za $f \in \A/\B_{\rr}$, ali $f: \Omega \rightarrow \C$ z $\set{\Re(f), \Im(f)} \subset L^1(\P)$, postavimo
$$\P[f \mid A] ~:=~ \frac{\P[f;A]}{\P(A)},$$
\textit{upanje $f$ pogojno na $A$}.

\end{definicija}
\vspace{0.5cm}

\begin{trditev}\label{ref:3.10}

Naj bo $(\Omega, \A, \P)$ verjetnostni prostor in $\BB$ pod-$\sigma$-algebra $\A$, $f \in \A/\B_{\rr}$, $\P[f^+],\P[f^-]<\infty$. Potem obstaja do $\P$-s.g. enakosti enoličen $g \in \BB/\B_{\rr}$ z $\P[g^+]\wedge\P[g^-]<\infty$, tak da je
$$\P[f;B] ~=~ \P[g;B], \quad \forall B \in \BB,$$
t.j. $(f\cdot\P)\big|_\BB = g(\P\big|_\BB);$ ta $g$ je $\frac{d\oklepaj{(f\cdot\P)\big|_\BB}}{d\oklepaj{\P\big|_\BB}}$ s.g.-$\P$, če je $f\geq 0$ in $\P[f]<\infty$.

\end{trditev}
\vspace{0.5cm}

\begin{definicija}

$g$ iz trditve \ref{ref:3.10} označimo z 
$$\P[f \mid \BB] ~:=~ \P_\BB(f) ~:=~ \P_\BB f,$$ 
\textit{pogojno matematično upanje $f$ glede na $\BB$ pod $\P$}. Če je $\P[|f|]<\infty$ potem vztrajamo na temu, da je $|\P[f \mid \BB]|<\infty$ povsod. Za $A \in \A$ pišemo
$$\P(A \mid \BB) ~:=~ \P[\1_A \mid \BB] ~=:~ \P_\BB(A) ~=: \P_\BB A,$$
\textit{pogojna verjetnost glede na $\BB$ pod $\P$}. 

\end{definicija}
\vspace{0.5cm}

\begin{trditev}

Naj bo $(\Omega, \A, \P)$ verjetnostni prostor, $\BB$ pod-$\sigma$-algebra $\A$ in $f \in \A/\B_{\rr}$ z $\P[f^+]\wedge\P[f^-]<\infty$. Pogojno matematično upanje ima sledeče lastnosti:
\begin{enumerate}

\item[(i)] (\textit{Stabilnost.}) Če je $f \in \BB/\B_{\rr}$, potem je
$$\P[f \mid \BB] ~=~ f \quad \text{s.g.-}\P.$$

\item[(ii)] (\textit{Zakon popolne verjetnosti/stolpna lastnosti.})
$$\P[\P[f \mid \BB]] ~=~ \P[f]$$

\item[(iii)] Naj bo $\CC$ pod-$\sigma$-algebra $\A$.
\begin{itemize}
	\item[(a)] (\textit{Zaporedno pogojevanje/stolpna lastnost.}) Če sta $\BB$ in $\CC$ primerljiva glede na inkluzijo je
	$$\P[\P[f \mid \BB] \mid \CC] ~=~ \P[f \mid \BB \cap \CC] \quad \text{s.g.-}\P.$$
	\item[(b)] (\textit{Nerelevantnost trivialnih dogodkov.}) Če je $\BB\vee\P^{-1}(\set{0,1}) = \\=\CC\vee\P^{-1}(\set{0,1})$, potem je
	$$\P[f \mid \BB] ~=~ \P[f \mid \CC] \quad \text{s.g.-}\P.$$
\end{itemize}

\item[(iv)] (\textit{Nerelevantnost množic z mero nič.}) Če je $g \in \A/\B_{\rr}$, $\P[g^+]\wedge\P[g^-]<\infty$ in $g=f$ s.g.-$\P$ potem je 
$$\P[f \mid \BB] ~=~ \P[g \mid \BB] \quad \text{s.g.-}\P.$$

\item[(v)] (\textit{Pogojevanje na trivialno $\sigma$-algebro.}) Če je $\BB \subset \P^{-1}(\set{0,1})$, potem je
$$\P[f \mid \BB] ~=~ \P[f] \quad \text{s.g.-}\P.$$
V posebnem je $\P[f \mid \set{\emptyset,\infty}] = \P[f]$.

\item[(vi)] (\textit{Aditivnost.}) Če je še $g \in \A/\B_{\rr}$ in $\P[f^-]\vee\P[g^-]<\infty$, potem je
$$\P[f+g \mid \BB] ~=~ \P[f \mid \BB] + \P[g \mid \BB] \quad \text{s.g.-}\P.$$

\item[(vii)] Naj bo $\P[|f|]<\infty$. Naj bo še $g \in \BB/\B_{\rr}$ z $\P[|g|]<\infty$. Če je
$$\P[f;B] ~=~ \P[g;B], \quad \forall B \in \Pi \cup \set{\Omega}$$
za nek $\pi$-sistem $\Pi$, ki generira $\BB$ na $\Omega$, potem je
$$g ~=~ \P[f \mid \BB] \quad \text{s.g.-}\P.$$

\item[(viii)] Za $h \in \BB/\B_{\rr}$ z $\P[(hf)^+]\wedge\P[(hf)^-]<\infty$ je
$$\P[(h\P[f \mid \BB])^+]\wedge\P[(h\P[f \mid \BB])^-]<\infty$$ 
in
$$\P[fh] ~=~ \P[\P[f \mid \BB]h].$$

\item[(ix)] (\textit{Pogojni determinizen/homogenost.}) Če je $g \in \BB/\B_{\rr}$ z $\P[(fg)^+]\wedge\P[(fg)^-]<\infty$, potem je
$$\P[fg \mid \BB] ~=~ g\P[f \mid \BB] \quad \text{s.g.-}\P.$$
V posebnem za $c \in \R$ je $\P[cf \mid \BB] = c\P[f \mid \BB]$ s.g.-$\P$.

\item[(x)] (\textit{Monotonost.}) Če je $g \in \A/\B_{\rr}$ z $\P[g^-]\wedge\P[g^+]<\infty$ in če je $\P[f;B] \leq \P[g;B] ~\forall B \in \BB$ (v posebnem, če je $f\leq g$ s.g.-$\P$), potem je
$$\P[f \mid \BB] ~\leq~ \P[g \mid \BB] \quad \text{s.g.-}\P.$$

\item[(xi)] (\textit{Trikotniška neenakost.})
$$|\P[f \mid \BB]| ~\leq~ \P[|f| \mid \BB] \quad \text{s.g.-}\P$$

\item[(xii)] Za namene te točke opustimo predpostavko $\P[f^+]\wedge\P[f^-]<\infty$. Naj bo $(f_n)_{n \in \N}$ zaporedje v $\A/\B_{\rr}$ in $g \in \A/\B_{\rr}$ z $\P[g]<\infty$ in $g \geq f_n^-$ s.g.-$\P$, $\forall n \in \N$.
\begin{itemize}
	\item[(a)] (\textit{Monotona konvergenca.}) Če $f_n \uparrow f$, ko gre $n \rightarrow \infty$, s.g.-$\P$, potem tudi
	$$\P[f_n \mid \BB] ~\uparrow~ \P[f \mid \BB],$$
	ko gre $n \rightarrow \infty$, s.g.-$\P$.
	\item[(b)] (\textit{Fatou-jeva lema.})
	$\P[(\liminf_{n \rightarrow\infty}f_n)^-]<\infty$ in
	$$\liminf_{n \rightarrow \infty} \P[f_n \mid \BB] ~\geq~ \P[\liminf_{n \rightarrow \infty} f_n \mid \BB] \quad \text{s.p.-}\P.$$
	\item[(c)] (\textit{Dominirana konvergenca.}) Če je $|f_n| \leq g$ s.g.-$\P$, $\forall n \in \N$ in $f_n \xrightarrow{n \rightarrow \infty} f$ s.g.-$\P$, potem je $\P[[f]]<\infty$ in
	$$\P[f \mid \BB] ~=~ \lim_{n \rightarrow \infty} \P[f_n \mid \BB]$$
	s.g.-$\P$ in v $L^1(\P)$. 
\end{itemize}

\item[(xiii)] 
\begin{itemize}
	\item[(a)] (\textit{Neodvisno pogojevanje.}) Če je $\BB'$ še pod-$\sigma$-algebra $\A$, \hbox{$f' \in \A/\B_{\rr}$} z $\P[(f')^+]\wedge\P[(f')^-]<\infty$ in če je $\BB\vee\sigma(f)$ neodvisna od \hbox{$\BB'\vee\sigma(f')$} kot tudi $\P[(ff')^+]\wedge\P[(ff')^-]<\infty$, potem je
	$$\P[ff' \mid \BB\vee\BB'] ~=~ \P[f \mid \BB]\P[f' \mid \BB'] \quad \text{s.g.-}\P.$$
	\item[(b)] (\textit{Nerelevantnost neodvisnih dogodkov.}) Če je $\CC$ pod-$\sigma$-algebra $\A$, ki je neodvisna od $\BB\vee\sigma(f)$, potem je
	$$\P[f \mid \BB\vee\CC] ~=~ \P[f \mid \BB] \quad \text{s.g.-}\P.$$
	\item[(c)] (\textit{Pogojevanje na neodvisno $\sigma$-algebro.}) Če je $\BB$ neodvisna od $\sigma(f)$, potem je
	$$\P[f \mid \BB] ~=~ \P[f] \quad \text{s.p.-}\P.$$
\end{itemize}

\item[(xiv)] (\textit{Jensenova neenakost.}) Če je $\varphi: I \rightarrow \R$ konveksna, $I$ interval na $\R$, $f$ jemlje vrednosti v $I$, $\P[|f|]<\infty$, potem je $\P(\P[f \mid \BB] \in I) = 1$, $\varphi \circ f \in \A/\B_\R$, $\P[(\varphi \circ f)^-]<\infty$,
$$\P[\varphi \circ f \mid \BB] ~\geq~ \varphi \circ (\P[f \mid \BB]) \quad \text{s.g.-}\P$$
in $f = \P[f \mid \BB]$ s.g.-$\P$ na $\set{\P[f \mid \BB] \in \partial I}$.

\end{enumerate}

\end{trditev}
\vspace{0.5cm}

\begin{trditev}

Naj bo $(\Omega, \A, \P)$ verjetnostni prostor. Nekaj metod za direktno računanje matematičnega upanja:
\begin{enumerate}

\item[(i)] (\textit{Diskretno pogojno matematično upanje.}) Naj bo $I \in 2^\A$ števa particija $\Omega$. Potem je za $X \in \A/\B_{\rr}$
$$\P[X \mid \sigma_\Omega(\I)] ~=~ \sum_{I \in \I \cap \P^{-1}((0,1])} \1_I \cdot \P[X \mid I] \quad \text{s.p.-}\P,$$
torej $\P[X \mid \sigma_\Omega(\I)] = \P[X \mid I]$ na $I$, za vsak $I \in \I$ z $\P(I)>0$.

\item[(ii)] Naj bosta $(F,\F)$ in $(E,\EE)$ merljiva prostora, $X \in \A/\F$, $Y \in \A/\EE$ slučajna elementa, $h \in (\F\otimes\EE)/\B_{\rr}$, \hbox{$\P[h^+(X,Y)]\wedge\P[h^-(X,Y)]<\infty$.}
\begin{itemize}
	\item[(a)] (\textit{Absolutno zvezna slučajni elementa.}) Naj bosta $\f$ $\sigma$-končna mera na $(F,\F)$ in $\e$ $\sigma$-končna mera na $(E,\EE)$ in naj bo $\P_{(X,Y)} \ll \f \times \e$.\footnote{Zadnja predpostavka implicira $\P_X \ll \f$ in $\P_Y \ll \e$.} Označimo
	$$f_{1\,2} ~:=~ \frac{d\P_{(X,Y)}}{d(\f\times\e)}, \quad f_2 ~:=~ \frac{d\P_Y}{d\e}.$$ 
	Potem je $\P$-s.g.
	$$\P[h(X,Y) \mid \sigma^{\EE}(Y)] ~=~ c(Y),$$
	kjer je 
	$$c(y) ~:=~ \int h(x,y) \frac{f_{1\,2(x,y)}}{f_2(y)}\,\f(dx)\1_{\set{f_2>0}}y, \quad y \in \E.$$
	Naprej, $c \in \EE/\B_{\rr}$.
	\item[(b)] (\textit{Neodvisna slučajne elementa.}) Naj bo $\G$ pod-$\sigma$-algebra $A$, da je $Y \in \G/\EE$ in $X$ neodvisna od $\G$. Potem je $\P$-s.g.
	$$\P[h(X,Y) \mid \G] ~=~ d(Y),$$
	kjer je 
	$$d(y) := \P[h(X,y)], \quad y \in E.$$
	Naprej, $d \in \EE/\B_{\rr}$.
\end{itemize} 

\end{enumerate}

\end{trditev}
\vspace{0.5cm}

\begin{definicija}

Če je $(\Omega, \A, \P)$ verjetnostni prostr in $Z$ slučajni element z vrednostmi v merljivem prostoru $(E,\EE)$; potem pišemo
$$\P[f \mid Z] ~:=~ \P[f \mid \sigma^\EE(Z)]$$
za $f \in \A/\B_{\rr}$ z $\P[f^+]\wedge\P[f^-]<\infty$.

\end{definicija}
\vspace{0.5cm}

\begin{trditev}

Naj bo $(\Omega, \A, \P)$ verjetnostni prostor in $Z$ slučajni element z vrednostmi v merljivem prostoru $(E,\EE)$. Za $f \in \A/\B_{\rr}$ z $\P[f^+]\wedge\P[f^-]<\infty$ obstaja $\P_Z$-s.g. enolično določen $g \in \EE/\B_{\rr}$, da je
$$\P[f \mid Z] ~=~ g(Z) \quad \text{s.g.-}\P.$$

\end{trditev}
\vspace{0.5cm}

% *************************************************************************************************

\pagebreak

% #################################################################################################

\end{document}